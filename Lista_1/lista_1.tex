\documentclass[12 pt]{article}
\usepackage[utf8]{inputenc}
\usepackage{matlab-prettifier}
\usepackage[portuguese]{babel}
\usepackage{indentfirst}
\usepackage{graphicx}
\usepackage{float}
\usepackage{subcaption}
\usepackage[font=small,labelfont=bf]{caption}
\definecolor{mygreen}{RGB}{28,172,0} % color values Red, Green, Blue
\definecolor{myyellow}{rgb}{1.0, 1.0, 0.8}
\usepackage{mathtools}
\usepackage{multirow}
\usepackage{comment}
\usepackage{xcolor}
\usepackage{colortbl}
\usepackage[normalem]{ulem}               % to striketrhourhg text
\usepackage{amsmath}
\usepackage{amsfonts}
\usepackage{hyperref}
\usepackage{tcolorbox}
\usepackage{longtable}
\usepackage{enumitem}
\newcommand\redout{\bgroup\markoverwith
{\textcolor{red}{\rule[0.5ex]{2pt}{0.8pt}}}\ULon}
\renewcommand{\lstlistingname}{Código}% Listing -> Algorithm
\renewcommand{\lstlistlistingname}{Lista de \lstlistingname s}% List of Listings -> List of Algorithms

\usepackage[top=3cm,left=2cm,bottom=2cm, right=2cm]{geometry}
\usepackage{tikz}
\usetikzlibrary{decorations.pathreplacing}
\usetikzlibrary{automata}
\usetikzlibrary{positioning}
\usetikzlibrary{arrows.meta, positioning}

\usepackage{adjustbox}


% Configuração para destacar a sintaxe do Python
\lstset{ 
    language=Python,                     % A linguagem do código
    backgroundcolor=\color{myyellow}, % A cor do fundo 
    basicstyle=\ttfamily\footnotesize,   % O estilo do texto básico
    keywordstyle=\color{blue},           % Cor das palavras-chave
    stringstyle=\color{red},             % Cor das strings
    commentstyle=\color{mygreen},          % Cor dos comentários
    numbers=left,                        % Números das linhas à esquerda
    numberstyle=\tiny\color{gray},       % Estilo dos números das linhas
    stepnumber=1,                        % Número de linhas entre os números das linhas
    frame=single,                        % Moldura ao redor do código
    breaklines=true,                     % Quebra automática das linhas longas
    captionpos=t,                        % Posição da legenda
    showstringspaces=false               % Não mostra espaços em branco nas strings
    extendedchars=true,
    literate={º}{{${ }^{\underline{o}}$}}1 {á}{{\'a}}1 {à}{{\`a}}1 {ã}{{\~a}}1 {é}{{\'e}}1 {É}{{\'E}}1 {ê}{{\^e}}1 {ë}{{\"e}}1 {í}{{\'i}}1 {ç}{{\c{c}}}1 {Ç}{{\c{C}}}1 {õ}{{\~o}}1 {ó}{{\'o}}1 {ô}{{\^o}}1 {ú}{{\'u}}1 {â}{{\^a}}1 {~}{{$\sim$}}1
}


\title{%
\textbf{\huge Universidade Federal do Rio de Janeiro} \par
\textbf{\LARGE Instituto Alberto Luiz Coimbra de Pós-Graduação e Pesquisa de Engenharia} \par

\includegraphics[width=8cm]{COPPE UFRJ.png} \par

\textbf{Programa de Engenharia de Sistemas e Computação} \par

CPS767 - Algoritmos de Monte Carlo e Cadeias de Markov  \par

Prof. Daniel Ratton Figueiredo\par

\vspace{1\baselineskip}
\textbf{\textit{1ª Lista de Exercícios}} \par
}

\author{Luiz Henrique Souza Caldas\\email: lhscaldas@cos.ufrj.br}

\date{\today}

\begin{document}
\maketitle

% \tableofcontents

\section*{Questão 1: Filhos e filhas}
Considere um casal que tem dois descendentes e que as chances de cada um deles ser filho ou filha são
iguais. Responda às perguntas abaixo:

\begin{enumerate}
    \item Calcule a probabilidade dos descendentes formar um casal (ou seja, um filho e uma filha).
    \begin{tcolorbox}[colframe=black, title=Resposta:]
        Seja $X_1$ a variável aleatória que representa o sexo do primeiro filho e $X_2$ a variável aleatória que representa o sexo do segundo filho, as quais assumem valor 0 se for uma filha e 1 se for um filho. O espaço amostral é dado por todas as combinações possíveis de sexos para os dois filhos, ou seja, $\Omega = \{(0,0), (0,1), (1,0), (1,1)\}$. Já o evento ``Formar um casal'' ocorre quando há um filho e uma filha, ou seja, $C = \{(0,1), (1,0)\}$. Como há $|C| = 2$ elementos no evento de interesse e $|\Omega| = 4$ no total, a probabilidade de formar um casal é dada por:
        $$P(C) = \frac{|C|}{|\Omega|} = \frac{2}{4} = \underline{ \frac{1}{2}\quad \vline}$$
    \end{tcolorbox}

    \item Calcule a probabilidade de ao menos um dos descendentes ser filho.
    \begin{tcolorbox}[colframe=black, title=Resposta:]
        Novamente espaço amostral é dado por todas as combinações possiveis de filhos e filhas, ou seja, $\Omega = \{(0,0), (0,1), (1,0), (1,1)\}$. O evento ``Ao menos um filho'' ocorre quando há um filho e uma filha ou dois filhos, ou seja, $F = \{(0,1), (1,0), (1,1)\}$. Como há $|F| = 3$ elementos no evento de interesse e $|\Omega| = 4$ no total, a probabilidade de ao menos um dos descendentes ser filho é dada por:
        $$P(F) = \frac{|F|}{|\Omega|} = \underline{\frac{3}{4} \quad \vline}$$
    \end{tcolorbox}
    
    \item Calcule a probabilidade das duas serem filhas dado que uma é filha (cuidado!).
    \begin{tcolorbox}[colframe=black, title=Resposta:]
        Se uma já é filha, o espaço amostral é reduzido para $\Omega_F = \{(0,1), (1,0), (0,0)\}$ e o evento de interesse é $D = \{(0,0)\}$. Como há $|D| = 1$ elemento no evento de interesse e $|\Omega_F| = 3$ no total, a probabilidade das duas serem filhas dado que uma é filha é dada por:
        $$P(D|F) = \frac{|D|}{|F|} = \underline{ \frac{1}{3} \quad \vline}$$
    \end{tcolorbox}
    \newpage
    \item Calcule a probabilidade dos descendentes nascerem no mesmo dia (assuma que a chance de nascer em um determinado dia é igual a qualquer outro).
    \begin{tcolorbox}[colframe=black, title=Resposta:]
        Supondo que o ano não seja bisexto, a probabilidade de nascer em um determinado dia é $\frac{1}{365}$. Como os nascimentos são independentes, a probabilidade dos descendentes nascerem no mesmo dia é dada por:
        $$P(\text{Mesmo dia}) = \left(\frac{1}{365}\right)^2 = \underline{ \frac{1}{133225} \quad \vline}$$
    \end{tcolorbox}
\end{enumerate}

\section*{Questão 2: Dado}
Considere um icosaedro (um sólido Platônico de 20 faces) tal que a chance de sair a face $i = 1, \dots, 20$ seja
linearmente proporcional a $i$. Ou seja, $P [X = i] = ci$ para alguma constante $c$, onde $X$ é uma variável
aleatória que denota a face do dado. Responda às perguntas abaixo:

\begin{enumerate}
    \item Determine o valor de $c$.
    \begin{tcolorbox}[colframe=black, title=Resposta:]
        A soma da probabilidade de sair cada face do dado deve ser igual a 1, ou seja:
        $$\sum_{i=1}^{20} P [X = i] = \sum_{i=1}^{20} ci = 1 \Rightarrow c\sum_{i=1}^{20} i = 1 \Rightarrow c\frac{20 \cdot 21}{2} = 1 \Rightarrow c = \underline{ \frac{1}{210} \quad \vline}$$
    \end{tcolorbox}
    \item Calcule o valor esperado de $X$ (obtenha também o valor numérico).
    \begin{tcolorbox}[colframe=black, title=Resposta:]
        O valor esperado de uma variável aleatória discreta é dado por:
        $$E[X] = \sum_{i=1}^{20} i \cdot P [X = i] = \sum_{i=1}^{20} i \cdot \frac{i}{210} = \frac{1}{210} \sum_{i=1}^{20} i^2 = \frac{1}{210} \cdot \frac{20 \cdot 21 \cdot 41}{6} = \frac{41}{3} \approx \underline{13,67 \quad \vline}$$
    \end{tcolorbox}
    \newpage
    \item Calcule a probabilidade de $X$ ser maior do que seu valor esperado.
    \begin{tcolorbox}[colframe=black, title=Resposta:]
        Como o valor esperado de $X$ é $E[X] = 13,67$, mas $X$ só pode assumir valores inteiros, a probabilidade de $X$ ser maior do que seu valor esperado é a mesma de $X$ ser maior ou igual a 14., dada por:
        $$P [X > E[X]] = P [X \geq 14] =  \sum_{i=14}^{20} \frac{i}{210} $$
        $$P [X > E[X]] = \frac{14+15+16+17+18+19+20}{210} \approx \underline{0,5667 \quad \vline}$$

    \end{tcolorbox}
    \item Calcule a variância de $X$ (obtenha também o valor numérico).
    \begin{tcolorbox}[colframe=black, title=Resposta:]
        A variância de uma variável aleatória discreta é dada por:
        $$\text{Var}[X] = E[X^2] - E[X]^2$$
        Onde $E[X^2]$ é dado por:
        $$E[X^2] = \sum_{i=1}^{20} i^2 \cdot P [X = i] = \sum_{i=1}^{20} i^2 \cdot \frac{i}{210} = \frac{1}{210} \sum_{i=1}^{20} i^3 = \frac{1}{210} \cdot \left(\frac{20 \cdot 21}{2}\right)^2 = 210$$
        Portanto, a variância de $X$ é dada por:
        $$\text{Var}[X] = 210 - \left(\frac{41}{3}\right)^2 \approx 210 - 186,77 = \underline{23,23 \quad \vline}$$
    \end{tcolorbox}
    \newpage
    \item Repita os últimos três itens para o caso do dado ser uniforme, ou seja, $P [X = i] = \frac{1}{20}$, $i = 1, \dots, 20$. Qual dado possui maior variância? Explique intuitivamente sua descoberta.
    \begin{tcolorbox}[colframe=black, title=Resposta:]
        $$E[X] = \sum_{i=1}^{20} i \cdot \frac{1}{20} = \frac{21}{2} = \underline{10,5 \quad \vline}$$
        $$P [X > E[X]] = \sum_{i=11}^{20} \frac{1}{20} = \frac{10}{20} = \frac{1}{2} = \underline{0,5 \quad \vline}$$
        $$E[X^2] = \sum_{i=1}^{20} i^2 \cdot \frac{1}{20} = \frac{20 \cdot 21 \cdot 41}{20 \cdot 6} = \frac{287}{2}$$
        $$\text{Var}[X] = E[X^2] - E[X]^2 = \frac{287}{2} - \left(\frac{21}{2}\right)^2 = \frac{287}{2} - \frac{441}{4} = \frac{133}{4} \approx \underline{33,25 \quad \vline}$$
        O dado com $P[X = i] = ci$ possui probabilidades maiores para valores maiores de $i$, concentrando mais a distribuição em torno de valores maiores. Isso faz com que o valor esperado seja maior e que a variância seja menor em comparação ao dado com distribuição uniforme. Portanto, \underbar{o dado com distribuição uniforme} $P[X = i] = \frac{1}{20}$ \underbar{possui maior variância}.
    \end{tcolorbox}
\end{enumerate}

\section*{Questão 3: Dado em ação}
Considere a versão uniforme do dado acima, ou seja, $P [X = i] = \frac{1}{20}$, $i = 1, \dots, 20$. Seja $Y$ uma variável
aleatória indicadora da primalidade da face do dado. Ou seja, $Y = 1$ quando o $X$ é um número primo,
e $Y = 0$ caso contrário. Responda às perguntas abaixo:

\begin{enumerate}
    \item Determine $P [Y = 1]$.
    \begin{tcolorbox}[colframe=black, title=Resposta:]
        Seja $U$ o espaço amostral com todos os valores possíveis de $X$, ou seja, $U = \{1, 2, 3, \dots, 20\}$. Seja $C=\{2, 3, 5, 7, 11, 13, 17, 19\}$ o evento com os números primos em $U$. Como os lançamentos são intependentes e a probabilidade de sair um número primo é $P[x_i] = \frac{1}{20}, \quad \forall i \in U$, a probabilidade de $Y = 1$ é dada por:
        $$P[Y = 1] = \sum_{j \in C} P[X = j] = \sum_{j \in C} \frac{1}{20} = \frac{|C|}{20} = \frac{8}{20} = \underline{\frac{2}{5} \quad \vline}$$
    \end{tcolorbox}
    \newpage
    \item Considere que o dado será jogado $n$ vezes. Seja $Y_i$ a indicadora da primalidade da $i$-ésima rodada, para $i = 1, \dots, n$, e defina $Z = \sum_{i=1}^{n} Y_i$. Repare que $Z$ é uma variável aleatória que denota o número de vezes que o resultado é primo. Determine a distribuição de $Z$, ou seja, $P [Z = k]$, para $k = 0, \dots, n$. Que distribuição é esta?
    \begin{tcolorbox}[colframe=black, title=Resposta:]
        Seja $S=\{Y_1=0, Y_2=1, ..., Y_n=1\}$ uma sequêcia de $n$ lançamentos de dado, com $k$ sucessos (números primos) e $n-k$ fracassos (números não primos). Se os lançamentos são independentes, a probabilidade de $k$ sucessos em $n$ tentativas é dada por:
        $$P[S] = \left(\prod_{i=1}^{n} P[Y_i] \cdot \mathbb{I}(Y_i = 1) \right) \left(\prod_{i=1}^{n} P[Y_i] \cdot \mathbb{I}(Y_i = 0) \right),$$
        onde $\mathbb{I}(Y_i = 1)$ é a função indicadora que vale 1 se $Y_i = 1$ e 0 caso contrário. Como $P[Y_i = 1] = \frac{2}{5}$ e $P[Y_i = 0] = \frac{3}{5}$, a probabilidade de $k$ sucessos em $n$ tentativas é dada por:
        $$P[S] = \left(\frac{2}{5}\right)^k \left(\frac{3}{5}\right)^{n-k}$$.
        Esta é a probabilidade para uma sequência específica $S$ de $k$ sucessos e $n-k$ fracassos. Como há $\binom{n}{k}$ maneiras de escolher $k$ sucessos em $n$ tentativas, a probabilidade de $Z = k$ é dada por:
        $$\boxed{P[Z = k] = \binom{n}{k} \left(\frac{2}{5}\right)^k \left(\frac{3}{5}\right)^{n-k}}$$

        Por $Z$ representar um somatório de $n$ variáveis aleatórias de Bernoulli $Y_i$, a distribuição de $Z$ é, por definição, uma \underline{distribuição Binomial}.
    \end{tcolorbox}
    \newpage
    \item Considere que o dado será jogado até que um número primo seja obtido. Seja $Y_i$ a indicadora da primalidade da $i$-ésima rodada, para $i = 1, \dots$, e defina $Z = \min\{i | Y_i = 1\}$. Repare que $Z$ denota o número de vezes que o dado é jogado até que o resultado seja um número primo. Determine a distribuição de $Z$, ou seja $P [Z = k]$, para $k = 1, \dots$. Que distribuição é esta?
    \begin{tcolorbox}[colframe=black, title=Resposta:]
        Seja $S=\{Y_1=0, Y_2=0, ..., Y_{k-1}=0, Y_k=1\}$ uma sequência de $k$ lançamentos de dado, com $k-1$ fracassos (números não primos) e 1 sucesso (número primo), na última posição. Como os lançamentos são independentes, a probabilidade de $S$ é dada por:
        $$P[S] = \left(\prod_{i=1}^{k-1} P[Y_i = 0] \right)  P[Y_k=1]$$
        $$P[S] = \left(\frac{3}{5}\right)^{k-1} \frac{2}{5}$$
        Pela definição do enunciado, a sequencia $S$ é a única possível, então:
        $$\boxed{P[Z = k] = \left(\frac{3}{5}\right)^{k-1} \frac{2}{5}}$$
        Por $Z$ representar o número de tentativas até o primeiro sucesso, a distribuição de $Z$ é, por definição, uma \underline{distribuição Geométrica}.
    \end{tcolorbox}
\end{enumerate}

\section*{Questão 4: Cobra}
Considere três imagens tiradas em uma floresta, $I_1, I_2$, e $I_3$. Em apenas uma das imagens existe uma
pequena cobra. Um algoritmo de detecção de cobras em imagens detecta a cobra na imagem $i$ com
probabilidade $\alpha_i$. Suponha que o algoritmo não encontrou a cobra na imagem $I_1$. Defina o espaço
amostral e os eventos apropriados e use a regra de Bayes para determinar:

\newpage
\begin{enumerate}
    \item A probabilidade da cobra estar na imagem $I_1$.
    \begin{tcolorbox}[colframe=black, title=Resposta:]
        Seja $D_i$ o evento de o algoritmo detectar a cobra na imagem $I_i$, e $\bar{D}_i$ o evento complementar (o algoritmo não detectar a cobra na imagem $I_i$). Seja $C_i$ o evento de a cobra estar na imagem $I_i$, e $\bar{C}_i$ o evento complementar (a cobra não estar na imagem $I_i$). O problema pede $P[C_1 | \bar{D}_1]$, ou seja, a probabilidade de a cobra estar na imagem $I_1$, dado que o algoritmo não a detectou nessa imagem (Falso Negativo). Pela regra de Bayes, temos:
        $$
        P[C_1 | \bar{D}_1] = \frac{P[\bar{D}_1 | C_1]P[C_1]}{P[\bar{D}_1]}
        $$
        Onde:
        \begin{itemize}
            \item $P[\bar{D}_1 | C_1] = 1 - \alpha_1$ é a probabilidade de o algoritmo não detectar a cobra na imagem $I_1$, dado que a cobra está em $I_1$;
            \item $P[C_1] = P[C_2] = P[C_3] = \frac{1}{3}$ é a probabilidade a priori de a cobra estar na imagem $I_1$, assumindo que ela está em uma, e apenas uma, das três imagens com igual probabilidade;
            \item $P[\bar{D}_1]$ é a probabilidade total de o algoritmo não detectar a cobra na imagem $I_1$, calculada por $ P[\bar{D}_1] = \sum_{i=1}^{3} P[\bar{D}_1 | C_i]P[C_i]$.
        \end{itemize}

        Sabemos que:
        \begin{itemize}
            \item Se a cobra está em $I_1$, $P[\bar{D}_1 | C_1] = 1 - \alpha_1$;
            \item Se a cobra está em $I_2$ ou $I_3$, o algoritmo não pode detectá-la em $I_1$, pois ela não está lá (assumindo que não há Falsos Positivos). Portanto, $P[\bar{D}_1 | C_2] = P[\bar{D}_1 | C_3] = 1$.
        \end{itemize}

        Substituindo esses valores, temos:
        $$
        P[\bar{D}_1] = (1 - \alpha_1) \cdot \frac{1}{3} + 1 \cdot \frac{1}{3} + 1 \cdot \frac{1}{3} = \frac{3 - \alpha_1}{3}
        $$

        Portanto, a probabilidade de a cobra estar na imagem $I_1$, dado que o algoritmo não a detectou nessa imagem, é:
        $$
        P[C_1 | \bar{D}_1] = \frac{P[\bar{D}_1 | C_1]P[C_1]}{P[\bar{D}_1]} = \frac{(1 - \alpha_1) \cdot \frac{1}{3}}{\frac{3 - \alpha_1}{3}} $$

        $$\boxed{P[C_1 | \bar{D}_1] = \frac{1 - \alpha_1}{3 - \alpha_1}}$$
    \end{tcolorbox}
    \newpage
    \item A probabilidade da cobra estar na imagem $I_2$.
    \begin{tcolorbox}[colframe=black, title=Resposta:]
        Este item pede $P[C_2 | \bar{D}_1]$, ou seja, a probabilidade de a cobra estar na imagem $I_2$, dado que o algoritmo não a detectou na imagem $I_1$. Pela regra de Bayes, temos:
        $$
        P[C_2 | \bar{D}_1] = \frac{P[\bar{D}_1 | C_2]P[C_2]}{P[\bar{D}_1]}
        $$
        Como todos os termos já foram calculados no item anterior, temos:
        $$
        P[C_2 | \bar{D}_1] = \frac{1 \cdot \frac{1}{3}}{\frac{3 - \alpha_1}{3}} = \frac{1}{3 - \alpha_1}
        $$
        $$\boxed{P[C_2 | \bar{D}_1] = \frac{1}{3 - \alpha_1}}$$
    \end{tcolorbox}
\end{enumerate}

\section*{Questão 5: Sem memória}
Seja $X \sim \text{Geo}(p)$ uma variável aleatória Geométrica com parâmetro $p$. Mostre que a distribuição
geométrica não tem memória. Ou seja, dado que $X > k$, o número de rodadas adicionais até que o evento
de interesse ocorra possui a mesma distribuição (dica: formalize esta afirmação).
\begin{tcolorbox}[colframe=black, title=Resposta:]
Se $X \sim \text{Geo}(p)$ é uma variável aleatória Geométrica com parâmetro $p$, temos:
$$P[X > k] = \prod_{i=1}^k (1-p) = (1-p)^k$$
Ou seja, $k$ fracassos seguidos em uma sequência de variáveis aleatórias de Bernoulli, pois o restante da sequência para $i>k$ é irrelevante. Seja $m$ um número inteiro positivo, pela regra do produto, temos:
$$P[X > k + m | X > k] = \frac{P[(X > k + m) \cap (X > k)]}{P[X > k]}$$
Porém $m$ e $k$ são inteiros positivos, então $P[(X > k + m) \cap (X > k)] = P[X > k + m]$. Portanto:
$$P[X > k + m | X > k] = \frac{P[X > k + m]}{P[X > k]} = \frac{(1-p)^{k+m}}{(1-p)^k} = (1-p)^m$$ 
$$ \boxed{P[X > k + m | X > k] = (1-p)^m = P[X > m]}$$
\end{tcolorbox}

\section*{Questão 6: Ônibus}
Considere que o processo de chegada do ônibus 485 no ponto do CT seja bem representado por um
processo de Poisson. Ou seja, $X \sim \text{Poi}(\lambda, t)$ denota o número (aleatório) de ônibus que chegam ao ponto
em um intervalo de tempo $t$ com taxa média de chegada igual a $\lambda$. Assuma que $\lambda = 10$ ônibus por hora.

\begin{enumerate}
    \item Determine a probabilidade de não chegar nenhum ônibus em um intervalo de 30 minutos (inclusive numericamente).
    \begin{tcolorbox}[colframe=black, title=Resposta:]
        A distribuição de Poisson é dada por $$ P[X = k | \lambda, t] = \frac{e^{-\lambda t}(\lambda t)^k}{k!}$$
        Para $k=0$, $\lambda = 10$ ônibus por hora e $t = 0.5$ horas (30 minutos), temos: 

        $$P[X = 0 | 10, 0.5] = \frac{e^{-10 \cdot 0.5}(10 \cdot 0,5)^0}{0!} = e^{-5} \approx \underline{6.74 \times 10^{-3} \quad \vline}$$
    \end{tcolorbox}
    \item Determine a probabilidade da média ocorrer, ou seja, de chegarem exatamente 10 ônibus em uma hora (inclusive numericamente).
    \begin{tcolorbox}[colframe=black, title=Resposta:]
        Para $k=10$, $\lambda = 10$ ônibus por hora e $t = 1$ hora, temos: 

        $$ P[X = 10 | 10, 1] = \frac{e^{-10 \cdot 1}(10 \cdot 1)^{10}}{10!} = \frac{e^{-10} \cdot 10^{10}}{3628800} \approx \underline{0.1251 \quad \vline}$$
    

    \end{tcolorbox}
    \item Determine a taxa $\lambda$ tal que a probabilidade de chegar ao menos um ônibus em um intervalo de 5
          minutos seja maior que 90\% (inclusive numericamente).
          \begin{tcolorbox}[colframe=black, title=Resposta:]
            $$P[X>1 | \lambda, t] = 1 - P[X=0 | \lambda, t] = 1 - \frac{e^{-\lambda t}(\lambda t)^0}{0!} = 1 - e^{-\lambda t}$$
            Para $t = 1/12$ horas (5 minutos), temos:
            $$1 - e^{-\lambda \cdot 1/12} > 0.9 \Rightarrow e^{-\lambda \cdot 1/12} < 0.1 $$
            $$-\lambda \cdot 1/12 < \ln(0.1) \Rightarrow \lambda > 12 \ln(10) $$
            $$\boxed{\lambda > 27.63 \quad \text{ônibus/hora}}$$

          \end{tcolorbox}
\end{enumerate}

\section*{Questão 7: Propriedades}
Sejam $X$ e $Y$ duas variáveis aleatórias discretas. Mostre as seguintes equivalências usando as definições:

\begin{enumerate}
    \item $E[X] = E[E[X|Y]]$, conhecida como regra da torre da esperança.
    \begin{tcolorbox}[colframe=black, title=Resposta:]

    \end{tcolorbox}
    \item $\text{Var}[X] = E[X^2] - E[X]^2$.
    \begin{tcolorbox}[colframe=black, title=Resposta:]

    \end{tcolorbox}
\end{enumerate}

\section*{Questão 8: Paradoxo do Aniversário}
Considere um grupo com $n$ pessoas e assuma que a data de nascimento de cada uma é uniforme dentre
os 365 dias do ano. Vamos calcular a chance de duas ou mais pessoas fazerem aniversário no mesmo dia.

\begin{enumerate}
    \item Seja $c(n)$ a probabilidade de duas ou mais pessoas fazerem aniversário no mesmo dia. Determine explicitamente $c(n)$.
    \begin{tcolorbox}[colframe=black, title=Resposta:]

    \end{tcolorbox}
    \item Usando a aproximação $e^x \approx 1 + x$, determine o valor aproximado para $c(n)$.
    \begin{tcolorbox}[colframe=black, title=Resposta:]

    \end{tcolorbox}
    \item Usando o valor aproximado de $c(n)$, determine o menor valor de $n$ tal que a chance de colisão na data de aniversário seja maior do que $1/2$. Você considera este número alto ou baixo?
    \begin{tcolorbox}[colframe=black, title=Resposta:]

    \end{tcolorbox}
\end{enumerate}

\section*{Questão 9: Caras em sequência}
Considere uma moeda enviesada, tal que a probabilidade do resultado ser cara é $p$ (e coroa $1 - p$).
Considere o número de vezes que a moeda precisa ser jogada para obtermos $k$ caras consecutivas. Por exemplo, na sequência ``COCOCCOOCOCCC'' a moeda teve que ser jogada 13 vezes até o aparecimento de $k = 3$ caras consecutivas, onde C = cara e O = coroa.


\begin{enumerate}
    \item Seja $N_k$ a variável aleatória que denota esta quantidade. Qual é o número médio de vezes que a moeda precisa ser jogada para obtermos $k$ caras consecutivas, ou seja, qual é o valor esperado de $N_k$? Dica: comece com $k = 1$, monte uma recursão em $k$, e use a regra da torre da esperança.
    \begin{tcolorbox}[colframe=black, title=Resposta:]

    \end{tcolorbox}
\end{enumerate}

\section*{Códigos}

Os códigos utilizados para a resolução dos exercícios estão disponíveis no repositório do GitHub: \url{https://github.com/lhscaldas/CPS767_MCMC/}

% \bibliographystyle{abntex2-num} % Escolha o estilo de citação desejado
% \nocite{sutton2018reinforcement}
% \bibliography{bibliografia} % Nome do arquivo .bib (sem a extensão)

\end{document}