\documentclass[12 pt]{article}
\usepackage[utf8]{inputenc}
\usepackage{matlab-prettifier}
\usepackage[portuguese]{babel}
\usepackage{indentfirst}
\usepackage{graphicx}
\usepackage{float}
\usepackage{subcaption}
\usepackage[font=small,labelfont=bf]{caption}
\definecolor{mygreen}{RGB}{28,172,0} % color values Red, Green, Blue
\definecolor{myyellow}{rgb}{1.0, 1.0, 0.8}
\usepackage{mathtools}
\usepackage{multirow}
\usepackage{comment}
\usepackage{xcolor}
\usepackage{colortbl}
\usepackage[normalem]{ulem}               % to striketrhourhg text
\usepackage{amsmath}
\usepackage{amsfonts}
\usepackage{hyperref}
\usepackage{tcolorbox}
\usepackage{longtable}
\usepackage{enumitem}
\newcommand\redout{\bgroup\markoverwith
{\textcolor{red}{\rule[0.5ex]{2pt}{0.8pt}}}\ULon}
\renewcommand{\lstlistingname}{Código}% Listing -> Algorithm
\renewcommand{\lstlistlistingname}{Lista de \lstlistingname s}% List of Listings -> List of Algorithms

\usepackage[top=3cm,left=2cm,bottom=2cm, right=2cm]{geometry}
\usepackage{tikz}
\usetikzlibrary{decorations.pathreplacing}
\usetikzlibrary{automata}
\usetikzlibrary{positioning}
\usetikzlibrary{arrows.meta, positioning}

\usepackage{adjustbox}


% Configuração para destacar a sintaxe do Python
\lstset{ 
    language=Python,                     % A linguagem do código
    backgroundcolor=\color{myyellow}, % A cor do fundo 
    basicstyle=\ttfamily\footnotesize,   % O estilo do texto básico
    keywordstyle=\color{blue},           % Cor das palavras-chave
    stringstyle=\color{red},             % Cor das strings
    commentstyle=\color{mygreen},          % Cor dos comentários
    numbers=left,                        % Números das linhas à esquerda
    numberstyle=\tiny\color{gray},       % Estilo dos números das linhas
    stepnumber=1,                        % Número de linhas entre os números das linhas
    frame=single,                        % Moldura ao redor do código
    breaklines=true,                     % Quebra automática das linhas longas
    captionpos=t,                        % Posição da legenda
    showstringspaces=false               % Não mostra espaços em branco nas strings
    extendedchars=true,
    literate={º}{{${ }^{\underline{o}}$}}1 {á}{{\'a}}1 {à}{{\`a}}1 {ã}{{\~a}}1 {é}{{\'e}}1 {É}{{\'E}}1 {ê}{{\^e}}1 {ë}{{\"e}}1 {í}{{\'i}}1 {ç}{{\c{c}}}1 {Ç}{{\c{C}}}1 {õ}{{\~o}}1 {ó}{{\'o}}1 {ô}{{\^o}}1 {ú}{{\'u}}1 {â}{{\^a}}1 {~}{{$\sim$}}1
}


\title{%
\textbf{\huge Universidade Federal do Rio de Janeiro} \par
\textbf{\LARGE Instituto Alberto Luiz Coimbra de Pós-Graduação e Pesquisa de Engenharia} \par

\includegraphics[width=8cm]{COPPE UFRJ.png} \par

\textbf{Programa de Engenharia de Sistemas e Computação} \par

CPS767 - Algoritmos de Monte Carlo e Cadeias de Markov  \par

Prof. Daniel Ratton Figueiredo\par

\vspace{1\baselineskip}
\textbf{\textit{2ª Lista de Exercícios}} \par
}

\author{Luiz Henrique Souza Caldas\\email: lhscaldas@cos.ufrj.br}

\date{\today}

\begin{document}
\maketitle

% \tableofcontents

\section*{Questão 1: Cauda do dado}

Considere um icosaedro (um sólido Platônico de 20 faces) honesto, tal que a probabilidade associada a cada face é 1/20. Considere que o dado será lançado até que um número primo seja observado, e seja $Z$ a variável aleatória que denota o número de vezes que o dado é lançado. Responda às perguntas abaixo:

\begin{enumerate}
    \item Determine a distribuição de $Z$, ou seja $P[Z = k], k = 1, 2, \dots$. Que distribuição é esta?
    \item Utilize a desigualdade de Markov para calcular um limitante para $P[Z \geq 10]$.
    \item Utilize a desigualdade de Chebyshev para calcular um limitante para $P[Z \geq 10]$.
    \item Calcule o valor exato de $P[Z \geq 10]$ (dica: use probabilidade complementar). Compare os valores obtidos.
\end{enumerate}

\section*{Questão 2: Pesquisa eleitoral}

Você leu no jornal que uma pesquisa eleitoral com 1500 pessoas indicou que 40\% dos entrevistados prefere o candidato A enquanto 60\% preferem o candidato B. Determine a margem de erro desta pesquisa usando uma confiança de 95\%. O que você precisou assumir para calcular a margem de erro?

\section*{Questão 3: Sanduíches}

Você convidou 64 pessoas para uma festa e agora precisa preparar sanduíches para os convidados. Você acredita que cada convidado irá comer 0, 1 ou 2 sanduíches com probabilidades 1/4, 1/2 e 1/4, respectivamente. Assuma que o número de sanduíches que cada convidado irá comer é independente de qualquer outro convidado. Quantos sanduíches você deve preparar para ter uma confiança de 95\% de que não vai faltar sanduíches para os convidados?

\section*{Questão 4: Graus improváveis}

Considere o modelo de grafo aleatório de Erdős-Rényi (também conhecido por $G(n, p)$), onde cada possível aresta de um grafo rotulado com $n$ vértices ocorre com probabilidade $p$, independentemente. Responda às perguntas abaixo:

\begin{enumerate}
    \item Determine a distribuição do grau do vértice 1 (em função de $n$ e $p$).
    \item Determine o valor $\gamma$ (em função de $n$ e $p$) tal que com alta probabilidade ($1 - 1/n$) o grau observado no vértice 1 é menor ou igual a $\gamma$.
\end{enumerate}

\section*{Questão 5: Calculando uma importante constante}

Seja $X_i$ uma sequência i.i.d. de v.a. contínuas uniformes em $[0, 1]$. Seja $V$ o menor número $k$ tal que a soma das primeiras $k$ variáveis seja maior do que 1. Ou seja, $V = \min\{k \mid X_1 + \cdots + X_k \geq 1\}$.

\begin{enumerate}
    \item Escreva e implemente um algoritmo para gerar uma amostra de $V$.
    \item Escreva e implemente um algoritmo de Monte Carlo para estimar o valor esperado de $V$.
    \item Trace um gráfico do valor estimado em função do número de amostras. Para qual valor seu estimador está convergindo?
\end{enumerate}

\section*{Questão 6: Transformada inversa}

Mostre como o método da transformada inversa pode ser usado para gerar amostras de uma v.a. contínua $X$ com as seguintes distribuições:

\begin{enumerate}
    \item Distribuição exponencial com parâmetro $\lambda > 0$, cuja função densidade é dada por $f_X(x) = \lambda e^{-\lambda x}$, para $x \geq 0$.
    \item Distribuição de Pareto com parâmetros $x_0 > 0$ e $\alpha > 0$, cuja função densidade é dada por $f_X(x) = \frac{\alpha x_0^\alpha}{x^{\alpha + 1}}$, para $x \geq x_0$.
\end{enumerate}

\section*{Questão 7: Contando domínios na Web}

Quantos domínios web existem dentro da UFRJ? Mais precisamente, quantos domínios existem dentro do padrão de nomes \texttt{http://www.[a-z](k).ufrj.br}, onde $[a-z](k)$ é qualquer sequência de caracteres de comprimento $k$ ou menor? Construa um algoritmo de Monte Carlo para estimar este número.

\begin{enumerate}
    \item Descreva a variável aleatória cujo valor esperado está relacionado com a medida de interesse. Relacione analiticamente o valor esperado com a medida de interesse.
    \item Implemente o método de Monte Carlo para gerar amostras e estimar a medida de interesse. Para determinar o valor de uma amostra, você deve consultar o domínio gerado para determinar se o mesmo existe (utilize uma biblioteca web para isto).
    \item Assuma que $k = 4$. Seja $\hat{w}_n$ o valor do estimador do número de domínios após $n$ amostras. Trace um gráfico em escala semi-log (eixo-$x$ em escala log) de $\hat{w}_n$ em função de $n$ para $n = 1, \dots, 10^5$ (ou mais, se conseguir). O que você pode dizer sobre a convergência de $\hat{w}_n$?
\end{enumerate}

\section*{Questão 8: Rejection Sampling}

Considere o problema de gerar amostras de uma v.a. $X \sim \text{Binomial}(1000, 0.2)$.

\begin{enumerate}
    \item Descreva uma proposta simples de função de probabilidade para gerar amostras de $X$ usando Rejection Sampling. Calcule a eficiência dessa proposta.
    \item Lembrando que a distribuição Binomial tem a forma de sino, centrada em sua média, proponha outra função de probabilidade para gerar amostras de $X$ usando Rejection Sampling. Calcule a eficiência dessa proposta e compare com a eficiência acima. O que você pode concluir?
\end{enumerate}

\section*{Questão 9: Integração de Monte Carlo e Importance Sampling}

Considere a função $g(x) = e^{-x^2}$ e a integral de $g(x)$ no intervalo $[0, 1]$.

\begin{enumerate}
    \item Implemente um método de Monte Carlo simples para estimar o valor da integral.
    \item Intuitivamente, muitas amostras de $g(x)$ vão ter valores muito baixos. Dessa forma, utilize Importance Sampling para melhorar a qualidade do estimador do valor da integral. Em particular, utilize a função de densidade $h(x) = Ae^{-x}$ definida em $[0, 1]$ onde $A$ é o valor da constante de normalização. Mostre como gerar amostras de $h(x)$.
    \item Compare os dois métodos. Trace um gráfico do erro relativo de cada um dos estimadores em função do número de amostras. Ou seja, $|\hat{I}_n - I|/I$ onde $I$ é o valor exato da integral e $\hat{I}_n$ é o valor do estimador com $n$ amostras, para $n = 10^1, 10^2, \dots, 10^6$.
\end{enumerate}


\section*{Códigos}

Os códigos utilizados para a resolução dos exercícios estão disponíveis no repositório do GitHub: \url{https://github.com/lhscaldas/CPS767_MCMC/}

% \bibliographystyle{abntex2-num} % Escolha o estilo de citação desejado
% \nocite{sutton2018reinforcement}
% \bibliography{bibliografia} % Nome do arquivo .bib (sem a extensão)

\end{document}