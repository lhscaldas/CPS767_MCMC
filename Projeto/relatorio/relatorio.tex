\documentclass[12pt]{article}

\usepackage{sbc-template}
\usepackage{graphicx,url}
\usepackage[utf8]{inputenc}
\usepackage[brazil]{babel}

\usepackage{float} % para usar [H] nas figuras
\usepackage{mathtools} % para usar \text em $$
\usepackage{subcaption} % para subfigure
     
\sloppy

\title{Planejamento Dinâmico de Rotas para VANTs em Patrulha Naval}

\author{Luiz Henrique Souza Caldas\inst{1}, Daniel Ratton Figueiredo\inst{1}} 


\address{Programa de Engenharia de Sistemas e Computação (PESC) \\ Instituto Alberto Luiz Coimbra de Pós-Graduação e Pesquisa de Engenharia (COPPE) \\ Universidade Federal do Rio de Janeiro (UFRJ)\\
  \email{lhscaldas@cos.ufrj.br, daniel@cos.ufrj.br}
}

\begin{document} 

\maketitle

% \begin{abstract}
% This work presents a dynamic route planning approach for Unmanned Aerial Vehicles (UAVs) in maritime patrol missions, considering the progressive detection of targets during flight. Three navigation policies are compared: \textit{passive}, which follows a fixed route; \textit{greed}, which prioritizes nearby targets; and \textit{Simulated Annealing}, which uses stochastic sampling to optimize the visitation order. Simulations are conducted in a modular environment with randomly placed ships, and metrics such as distance traveled, detection rate, and execution time are evaluated. Results show that while the \textit{Simulated Annealing} policy performs better in sparse scenarios, the \textit{greed} policy is computationally more efficient and better suited for real-time re-planning in denser environments.
% \end{abstract}
     
\begin{resumo} 
Este trabalho apresenta uma abordagem de planejamento dinâmico de rotas para VANTs em patrulhas navais, considerando a detecção progressiva de alvos ao longo da missão. São comparadas três políticas de navegação: \textit{passiva}, com rota fixa; \textit{greed}, que prioriza alvos próximos; e \textit{Simulated Annealing}, que utiliza amostragem estocástica para otimizar a ordem de visita. As simulações, conduzidas em ambiente modular com geração aleatória de navios, avaliam métricas como distância percorrida, taxa de detecção e tempo de execução. Os resultados mostram que, embora a política \textit{Simulated Annealing} obtenha melhor desempenho com poucos navios, a \textit{greed} é mais eficiente computacionalmente e mais adequada para replanejamentos em tempo real.
\end{resumo}

% \section{Introdução}

% O Brasil possui uma extensa costa de 8{,}7 mil quilômetros, com 68 portos e uma faixa litorânea que concentra mais da metade da população e do PIB do país. Além disso, o país possui aproximadamente 4{,}5 milhões de quilômetros quadrados de águas jurisdicionais, onde se encontram recursos estratégicos como 95\% do petróleo e 83\% do gás natural nacionais, e por onde transitam cerca de 95\% do comércio exterior \cite{andrade_2021}.
% Nesse contexto, a crescente adoção de Veículos Aéreos Não Tripulados (VANTs) em operações de vigilância marítima exige métodos mais robustos de planejamento de rotas, especialmente em cenários com alvos parcialmente conhecidos e detecção sensorial sujeita a limitações práticas. Trabalhos anteriores mostram que a vigilância baseada em varredura aérea pode ser modelada como uma variação do Problema do Caixeiro Viajante (TSP), incorporando restrições operacionais como autonomia limitada, sensores com alcances distintos e alvos móveis ou parcialmente observáveis \cite{marlow_2007}.
% Técnicas de otimização como o \textit{Simulated Annealing} têm se mostrado eficazes em problemas de roteamento com grande espaço de busca e múltiplos mínimos locais, sendo uma escolha promissora para o replanejamento progressivo de trajetos em ambientes com inserção dinâmica de alvos \cite{kosmas_2012}.
% Mais recentemente, abordagens com replanejamento dinâmico em tempo real vêm sendo propostas para garantir a cobertura de alvos não detectados ou mal inspecionados, especialmente em aplicações com VANTs e sensores embarcados \cite{penicka_2017}. Motivado por esses avanços, este trabalho propõe uma formulação adaptada do TSP para missões de patrulha marítima com inserção condicional de novos alvos, levando em conta restrições de distância lateral, autonomia do VANT e alcance heterogêneo de sensores.


\section{Introdução}

O uso de Veículos Aéreos Não Tripulados (VANTs) em operações de vigilância marítima tem se expandido significativamente, impulsionado por desafios associados à extensão da costa brasileira e à importância estratégica de suas águas jurisdicionais, que concentram a maior parte do comércio exterior e das reservas nacionais de petróleo e gás natural \cite{andrade_2021}. Esses cenários exigem métodos eficientes de planejamento de rotas, sobretudo quando os alvos são parcialmente conhecidos e os sensores têm capacidades limitadas.

A tarefa pode ser modelada como uma variação do Problema do Caixeiro Viajante (TSP), incorporando restrições operacionais como autonomia limitada, sensores de diferentes alcances e alvos móveis ou parcialmente observáveis \cite{marlow_2007}. Técnicas como o \textit{Simulated Annealing} têm mostrado bom desempenho em problemas de roteamento com múltiplos mínimos locais, sendo úteis para o replanejamento dinâmico \cite{kosmas_2012}. Abordagens mais recentes propõem replanejamento em tempo real com base em dados sensoriais para garantir cobertura progressiva de alvos não detectados \cite{penicka_2017}.

Este trabalho propõe uma abordagem baseada em TSP adaptado para patrulha marítima com inserção progressiva de novos alvos detectados durante a missão, respeitando as limitações operacionais do VANT e explorando estratégias de navegação dinâmicas.

\begin{figure}[H]
    \centering
    \includegraphics[width=0.5\textwidth]{fig/vant.png}
    \caption{Patrulha naval realizada por aeronave.}
    \small
    \textbf{Fonte:} \cite{marlow_2007}.
\end{figure}

\section{Objetivo}

O objetivo deste trabalho é desenvolver uma metodologia para o planejamento dinâmico de rotas de VANTs em missões de vigilância marítima, considerando a detecção progressiva de alvos ao longo do percurso. A proposta visa adaptar o Problema do Caixeiro Viajante (TSP) a um contexto em que novos pontos de interesse são identificados durante a missão. A solução deve permitir o replanejamento eficiente da rota, de forma a maximizar a inspeção de alvos relevantes, respeitando os limites de autonomia da aeronave.
% \section{Metodologia}

% A abordagem proposta consiste em simular missões de vigilância marítima utilizando Veículos Aéreos Não Tripulados (VANTs), os quais percorrem uma rota pré-definida composta por linhas paralelas. Essa rota representa um padrão sistemático de patrulhamento e é gerada automaticamente com base em parâmetros como ponto inicial, largura da área, espaçamento entre linhas e número de passagens.

% Durante o voo, o VANT está equipado com dois sensores: um radar, responsável por detectar navios dentro de um raio de ação (50 milhas náuticas), e uma câmera de inspeção visual com alcance menor (20 milhas náuticas), utilizada para confirmar a identificação de alvos. Quando um navio entra na zona de alcance do radar, seu estado é atualizado para ``detectado''. Caso venha a entrar na área de alcance da câmera, ele é considerado ``inspecionado''.

% O VANT atualiza sua lista de destinos a serem visitados com base em diferentes políticas de decisão:

% \begin{itemize}
%     \item \textbf{Política passiva}: o VANT mantém sua rota original, ignorando completamente os alvos detectados e identificando apenas os navios que a câmera alcança sem que ele desvie da trajetória planejada.

%     \item \textbf{Política \textit{greed}}: os navios detectados e os waypoints ainda não percorridos da rota original são considerados simultaneamente, sendo reordenados com base na distância em relação à posição atual do VANT, priorizando os mais próximos em cada instante.
%     \item \textbf{Política \textit{simulated annealing}}: os navios detectados e os waypoints ainda não percorridos da rota original também são considerados simultaneamente, mas utilizando uma técnica de amostragem estocástica baseada em \textit{Markov Chain Monte Carlo (MCMC)}. A lógica do algoritmo se baseia em uma cadeia de Markov com distribuição estacionária proporcional a uma função de Boltzmann, permitindo explorar o espaço de permutações de forma mais ampla, em busca de uma sequência globalmente mais eficiente em termos de distância percorrida.
% \end{itemize}

% O algoritmo de \textit{Simulated Annealing} implementado parte de uma solução inicial aleatória, composta pela lista de navios detectados e waypoints restantes, obtida por uma permutação aleatória desses pontos. O objetivo é encontrar a melhor ordem de visita, minimizando a distância total percorrida pelo VANT a partir de sua posição atual. A cada iteração, uma nova solução vizinha é gerada invertendo a ordem dos pontos em um subintervalo aleatório da rota atual. Essa operação de vizinhança é simples, mas suficientemente expressiva para explorar o espaço de soluções e escapar de mínimos locais.

% Esse espaço de soluções pode ser interpretado como um grafo em que cada vértice representa uma permutação possível dos pontos a visitar, e as arestas conectam permutações que diferem por uma única inversão de subsegmento. Esse grafo é conexo e possui transições simétricas, o que garante que todas as soluções podem ser eventualmente alcançadas a partir de qualquer configuração inicial.

% A nova solução é aceita com probabilidade 1 se sua distância total $f(s')$ for menor que a da rota atual $f(s)$, onde $s$ é a rota atual e $s'$ a rota vizinha gerada, e $f(s)$ denota a distância total percorrida ao seguir a rota $s$. Caso contrário, a nova rota ainda pode ser aceita com probabilidade $e^{-\frac{f(s') - f(s)}{T}}$, onde $T$ é um parâmetro chamado temperatura, que vai decaindo ao longo do processo. Essa estratégia permite aceitar, no início da execução, soluções piores de forma controlada, favorecendo a diversidade de caminhos explorados e contribuindo para escapar de mínimos locais.

% A temperatura $T$ é atualizada ao longo do processo por meio de uma estratégia de resfriamento exponencial, seguindo a equação $T = T_0 \cdot \beta^t$, onde $T_0$ é a temperatura inicial, $\beta$ é um fator de decaimento ($0 < \beta < 1$), e $t$ representa a etapa atual. Em cada nível de temperatura, o algoritmo realiza um número fixo $N$ de perturbações da rota. Ao longo do processo, a melhor rota encontrada e seu respectivo custo são salvos sempre que superam o melhor valor anterior. Assim, ao final da execução, o algoritmo retorna a melhor sequência observada durante toda a busca, e não necessariamente a última solução gerada.

% % Apesar de sua flexibilidade, o algoritmo de \textit{Simulated Annealing} não está livre de limitações. Em particular, se a temperatura for reduzida rapidamente demais ao longo da execução, o processo pode ficar preso em mínimos locais, impedindo a descoberta de soluções mais eficientes. A escolha da estratégia de resfriamento adequada é, portanto, um fator crítico para o sucesso do método, sendo geralmente determinada por experimentação e ajuste fino para cada cenário.

% Considera-se que os navios permanecem estáticos durante a simulação, dada a alta velocidade relativa do VANT em comparação às embarcações. O voo é interrompido automaticamente quando a autonomia total é atingida ou quando não há mais destinos a serem visitados. Os dados de desempenho como número de alvos detectados e inspecionados, distância percorrida e tempo de execução são registrados ao final da missão.

\section{Metodologia}

A abordagem proposta consiste em simular missões de vigilância marítima com Veículos Aéreos Não Tripulados (VANTs) que percorrem rotas pré-definidas compostas por linhas paralelas, geradas a partir de parâmetros como ponto inicial, largura da área, espaçamento entre linhas e número de passagens.

Durante o voo, o VANT utiliza dois sensores: um radar com alcance de 50 milhas náuticas (MN), que detecta navios, e uma câmera de inspeção visual com alcance de 20 MN, que confirma sua identificação. Um navio detectado pelo radar tem seu estado atualizado para ``detectado'', e se estiver ao alcance da câmera, passa a ser ``inspecionado''.

A cada passo da missão, o VANT atualiza sua rota com base em uma das seguintes políticas de decisão:

\begin{itemize}
    \item \textbf{Política passiva}: mantém a rota original, inspecionando apenas os navios que entram no alcance da câmera sem alterar o trajeto.
    \item \textbf{Política \textit{greed}}: reordena dinamicamente os waypoints restantes e os navios detectados, priorizando os mais próximos da posição atual do VANT.
    \item \textbf{Política \textit{simulated annealing}}: aplica uma técnica estocástica baseada em \textit{Markov Chain Monte Carlo (MCMC)} para buscar uma sequência de visita aos waypoints restantes e aos navios detectados mais eficiente em termos de distância.
\end{itemize}

O algoritmo de \textit{Simulated Annealing} inicia com uma permutação aleatória dos navios detectados e waypoints remanescentes. Em cada iteração, uma nova rota é gerada invertendo a ordem de um subintervalo da rota atual, escolhendo dois pontos da rota atual de forma uniforme e invertendo a ordem dos pontos entre eles. O espaço de soluções é representado por um grafo conexo, onde vértices são permutações possíveis e arestas conectam soluções que diferem por uma inversão, formando uma cadeia de Markov irredutível, aperiódica e simétrica.

A nova rota $s'$ é aceita se sua distância total $f(s')$ for menor que a atual $f(s)$; caso contrário, ainda pode ser aceita com probabilidade $e^{-\frac{f(s') - f(s)}{T}}$, onde $T$ é a temperatura. Esse mecanismo permite explorar soluções subótimas no início do processo, ajudando a escapar de mínimos locais.

A temperatura $T$ decresce exponencialmente segundo $T = T_0 \cdot \beta^t$, com $T_0$ como temperatura inicial, $\beta$ como fator de decaimento ($0 < \beta < 1$) e $t$ representando a iteração atual. A cada nível de temperatura, são realizadas $N$ perturbações, salvando-se a melhor rota encontrada até então. O algoritmo retorna a melhor solução global observada, não necessariamente a última.

Assume-se que os navios estejam estáticos durante a simulação, dada a alta velocidade do VANT. A missão termina quando a autonomia é atingida ou não restam destinos a visitar. Dados como número de alvos detectados e inspecionados, distância percorrida e tempo de execução são registrados ao final da simulação.



% \section{Resultados}

% Nesta seção, são apresentadas comparações entre as três políticas de navegação descritas anteriormente: passiva, \textit{greed} e \textit{simulated annealing}. Nas simulações, o VANT percorre uma trajetória sistemática sobre a área de interesse (AI), sendo redirecionado dinamicamente de acordo com a política adotada a cada passo, conforme os alvos detectados.

% \subsection{Valores de Entrada}

% Os valores utilizados no modelo são:

% \begin{itemize}
%     \item Tamanho da AI: $300 \times 300$ milhas náuticas (MN)
%     \item Velocidade do VANT: 300 nós (MN / hora)
%     \item Alcance do radar (detecção): 50 MN
%     \item Alcance da câmera (inspeção): 20 MN
%     \item Autonomia: 2400 MN
%     \item Número de navios na AI: 10, 25, 50, 75, 100, 125, 150, 175, 200.
%     \item Distribuição espacial dos navios: uniformente aleatória.
%     \item Velocidade dos navios: parados, para simplificar a análise.
% \end{itemize}

% Os parâmetros utilizados especificamente na política de \textit{Simulated Annealing} são:

% \begin{itemize}
%     \item Temperatura inicial $T_0$: 10.0
%     \item Temperatura mínima $T_{\text{min}}$: $10^{-4}$
%     \item Fator de resfriamento $\beta$: 0.90
%     \item Número de perturbações por temperatura: 50
% \end{itemize}

% Para cada combinação de parâmetros, os experimentos são repetidos sobre 100 instâncias geradas aleatoriamente, o que totaliza $3 \times 9 \times 100 = 2700$ simulações. Os resultados são agregados por média e as métricas são apresentadas em valores absolutos (como distância percorrida e tempo de execução) ou normalizados em percentual (como proporção de navios detectados ou inspecionados), conforme apropriado para a análise.

\section{Resultados}

Nesta seção, comparam-se as políticas de navegação passiva, \textit{greed} e \textit{simulated annealing}, com base em simulações onde o VANT percorre uma trajetória sistemática sobre a área de interesse (AI), sendo redirecionado conforme a política adotada e os alvos detectados.

\subsection{Valores de Entrada}

As simulações consideram os seguintes parâmetros:

\begin{itemize}
    \item Área de interesse: $300 \times 300$ MN
    \item Velocidade do VANT: 300 nós
    \item Alcance dos sensores: radar (50 MN), câmera (20 MN)
    \item Autonomia do VANT: 2400 MN
    \item Quantidade de navios: 10 a 200 (incrementos de 25)
    \item Distribuição dos navios: aleatória e estática
\end{itemize}

Parâmetros específicos do \textit{Simulated Annealing}:

\begin{itemize}
    \item $T_0 = 10.0$, $T_{\text{min}} = 10^{-4}$
    \item Fator de resfriamento: $\beta = 0.90$
    \item Iretaçoes por ciclo: $N=50$
\end{itemize}

Cada cenário é repetido 100 vezes para as 3 políticas e 9 níveis de densidade, totalizando 2700 simulações. Os resultados são apresentados como médias, em valores absolutos (ex: distância e tempo) ou percentuais (ex: taxa de detecção/inspeção), conforme o caso.

% \subsection{Ambiente de simulação}

% A simulação foi implementada de forma modular, com dois componentes principais: o módulo \texttt{AmbienteMaritimo}, responsável por representar o cenário de patrulha e gerar os navios de forma aleatória, e o módulo \texttt{VANT}, que modela o comportamento da aeronave, incluindo sua trajetória, sensores e política de navegação. A separação entre ambiente e agente permite encapsular responsabilidades e facilita a extensão da lógica de simulação.

% A cada passo da simulação, o VANT executa um ciclo que envolve duas ações principais: movimentação e varredura sensorial. Primeiro, o VANT avança em direção ao próximo ponto definido pela sua política de navegação, respeitando sua velocidade e intervalo de tempo configurado, sem considerar a dinâmica da aeronave. A lógica de movimentação leva em conta a autonomia da aeronave: se o deslocamento planejado excede a distância restante permitida, a simulação é encerrada. 

% Após o deslocamento, o VANT verifica a presença de navios dentro de dois raios distintos: o do radar (detecção inicial) e o da câmera (inspeção visual). Os estados dos navios são atualizados conforme sua proximidade com o VANT: de \textit{nao\_detectado} para \textit{detectado} quando entram no alcance do radar, e de \textit{detectado} para \textit{inspecionado} quando também estão no raio da câmera.

% A política de navegação — \textit{passiva}, \textit{greed} ou \textit{Simulated Annealing} — define como os pontos a serem visitados são selecionados e ordenados. A \textit{passiva} segue uma rota fixa baseada em linhas paralelas; a \textit{greed} reordena os pontos (navios detectados e waypoints remanescentes) por proximidade ao VANT; e a \textit{Simulated Annealing} aplica uma metaheurística para minimizar a distância total do percurso. A política é avaliada e aplicada dinamicamente a cada passo, permitindo que o VANT adapte seu trajeto de forma progressiva conforme novos navios são detectados.

\subsection{Ambiente de Simulação}

A simulação foi implementada com dois módulos principais: \texttt{AmbienteMaritimo}, que gera o cenário e os navios aleatoriamente, e \texttt{VANT}, que modela a trajetória, sensores e política de navegação da aeronave. Essa separação facilita a manutenção e extensibilidade do sistema.

A cada passo, o VANT realiza duas ações: movimenta-se em direção ao próximo ponto definido pela política de navegação (respeitando sua velocidade e autonomia), e executa a varredura sensorial. Se o deslocamento previsto exceder a autonomia restante, a simulação é encerrada.

Em seguida, o VANT verifica a presença de navios nos raios do radar (50 MN) e da câmera (20 MN). O estado de um navio é atualizado de \textit{nao\_detectado} para \textit{detectado} ao entrar no alcance do radar, e para \textit{inspecionado} quando também estiver ao alcance da câmera.

A política de navegação define como os pontos (navios detectados e waypoints remanescentes) são ordenados. A política \textit{passiva} segue a rota fixa; a \textit{greed} prioriza os pontos mais próximos; e a \textit{Simulated Annealing} aplica uma metaheurística para minimizar a distância total. A política é avaliada dinamicamente, permitindo que o VANT ajuste sua rota conforme novos alvos são descobertos

% \subsection{Distância percorrida}

% A Figura~\ref{fig:distancia} apresenta a média da distância total percorrida pelo VANT ao longo da missão, conforme a política de navegação e a quantidade de navios. A política \textit{passiva} mantém trajetória constante, enquanto as políticas com replanejamento (\textit{greed} e \textit{Simulated Annealing}) aumentam progressivamente a distância total percorrida à medida que mais navios são introduzidos no cenário.

% \begin{figure}[H]
%     \centering
%     \includegraphics[width=0.7\textwidth]{fig/resultado_dis.png}
%     \caption{Média da distância percorrida por política e quantidade de navios.}
%     \label{fig:distancia}
% \end{figure}


% Observa-se que tanto a política \textit{greed} quanto a de \textit{Simulated Annealing} saturam a curva, atingindo a autonomia máxima do VANT a partir de 100 navios. Antes desse ponto de saturação, a política \textit{Simulated Annealing} percorre uma distância média menor do que a \textit{greed}, indicando que o método está encontrando soluções mais eficientes em termos de distância total percorrida.

\subsection{Distância Percorrida}

A Figura~\ref{fig:distancia} mostra a média da distância total percorrida pelo VANT conforme a política de navegação e a quantidade de navios. A política \textit{passiva} mantém uma trajetória fixa, enquanto \textit{greed} e \textit{Simulated Annealing} aumentam a distância percorrida à medida que mais navios são introduzidos.

\begin{figure}[H]
    \centering
    \includegraphics[width=0.7\textwidth]{fig/resultado_dis.png}
    \caption{Média da distância percorrida por política e quantidade de navios.}
    \label{fig:distancia}
\end{figure}

A partir de 100 navios, ambas as políticas ativas atingem o limite de autonomia do VANT. Antes disso, \textit{Simulated Annealing} apresenta menor distância média que \textit{greed}, indicando trajetórias mais eficientes.


% \subsection{Detecção de navios}

% A Figura~\ref{fig:detectados} mostra a média percentual de navios detectados ao longo da missão, considerando os que foram detectados ao menos uma vez, independentemente de terem sido inspecionados. A política \textit{passiva} apresenta uma taxa de detecção aproximadamente constante e, na maioria dos cenários, superior às demais políticas, exceto entre 50 e 125 navios, onde ocorre um cruzamento temporário de desempenho. Esse comportamento está relacionado ao fato de que, ao não se desviar da rota original, o VANT percorre toda a área de interesse, aumentando a probabilidade de que todos os navios dentro do alcance do radar sejam detectados.


% \begin{figure}[H]
%     \centering
%     \includegraphics[width=0.7\textwidth]{fig/resultado_det.png}
%     \caption{Média percentual de navios detectados por política e quantidade de navios.}
%     \label{fig:detectados}
% \end{figure}

% As políticas \textit{greed} e \textit{Simulated Annealing} apresentam desempenho semelhante em cenários com até 75 navios. A partir de cenários mais densos, a política \textit{greed} tende a apresentar maior taxa de detecção. A queda geral no desempenho dessas duas políticas, em cenários de 100 navios em diante pode, ser associada à limitação de autonomia do VANT, conforme indicado na Figura~\ref{fig:distancia}, que mostra a saturação da distância percorrida nos cenários com maior número de navios.

\subsection{Detecção de Navios}

A Figura~\ref{fig:detectados} apresenta a média percentual de navios detectados ao longo da missão, considerando aqueles identificados pelo radar ao menos uma vez. A política \textit{passiva} mantém uma taxa de detecção quase constante e, na maioria dos cenários, superior às demais, exceto entre 50 e 125 navios, onde há um cruzamento temporário de desempenho.

\begin{figure}[H]
    \centering
    \includegraphics[width=0.7\textwidth]{fig/resultado_det.png}
    \caption{Média percentual de navios detectados por política e quantidade de navios.}
    \label{fig:detectados}
\end{figure}

Esse desempenho da política \textit{passiva} se deve ao fato de ela cobrir toda a área de interesse, maximizando as chances de detecção por radar. Já \textit{greed} e \textit{Simulated Annealing} têm desempenho similar até 75 navios, mas em cenários mais densos a \textit{greed} tende a detectar mais. A queda geral de desempenho dessas políticas a partir de 100 navios está ligada à limitação de autonomia do VANT, conforme indicado na Figura~\ref{fig:distancia}.

% \subsection{Inspeção de navios}

% A Figura~\ref{fig:inspecionados} apresenta a média percentual de navios inspecionados em função da quantidade total de navios no cenário, para cada política de navegação considerada. Observa-se que a política \textit{passiva} mantém taxas de inspeção muito inferiores às políticas \textit{greed} e \textit{Simulated Annealing} em todos os cenários, o que é esperado, já que essas políticas incluem mecanismos de desvio da rota original para inspecionar alvos detectados.

% \begin{figure}[H]
%     \centering
%     \includegraphics[width=0.7\textwidth]{fig/resultado_ins.png}
%     \caption{Média percentual de navios inspecionados por política e quantidade de navios.}
%     \label{fig:inspecionados}
% \end{figure}

% Assim como na detecção de navios, as políticas \textit{greed} e \textit{Simulated Annealing} apresentam desempenho próximo em cenários com até 75 navios. Em cenários mais densos, a política \textit{greed} tende a alcançar uma maior taxa de inspeção. A redução gradual nas taxas médias pode ser atribuída à limitação de autonomia do VANT, que, ao atingir sua capacidade máxima de voo, encerra a missão antes de completar a inspeção de todos os alvos.

\subsection{Inspeção de Navios}

A Figura~\ref{fig:inspecionados} mostra a média percentual de navios inspecionados em função da quantidade de navios no cenário. A política \textit{passiva} apresenta taxas significativamente inferiores às de \textit{greed} e \textit{Simulated Annealing}, como esperado, por não desviar da rota para confirmar alvos detectados.

\begin{figure}[H]
    \centering
    \includegraphics[width=0.7\textwidth]{fig/resultado_ins.png}
    \caption{Média percentual de navios inspecionados por política e quantidade de navios.}
    \label{fig:inspecionados}
\end{figure}

As políticas ativas têm desempenho semelhante até 75 navios. Em cenários mais densos, \textit{greed} tende a inspecionar mais. A queda nas taxas médias deve-se à limitação de autonomia do VANT, que encerra a missão antes de visitar todos os alvos.

% \subsection{Tempo de execução}

% A Figura~\ref{fig:tempo_execucao} exibe o tempo médio de execução das simulações para cada política, em função do número de navios presentes no cenário. A política \textit{Simulated Annealing} apresenta tempo crescente com o aumento do número de navios, enquanto as políticas \textit{passiva} e \textit{greed} mantêm tempos praticamente constantes e significativamente inferiores.


% \begin{figure}[H]
%     \centering
%     \includegraphics[width=0.7\textwidth]{fig/resultado_tem.png}
%     \caption{Tempo médio de execução da simulação por política e quantidade de navios.}
%     \label{fig:tempo_execucao}
% \end{figure}

% O número de iterações do algoritmo de \textit{Simulated Annealing} é determinado pelos parâmetros $T_0$, $T_{\text{min}}$, $\beta$ e o número de perturbações. Porém cada iteração, é calculado o custo total da rota, somando-se as distâncias entre pontos consecutivos, o que ocorre com complexidade $O(n)$, sendo $n$ o número de pontos — composto pelos \textit{waypoints} restantes e pelos navios detectados não inspecionados. Isso resulta em um tempo de execução que cresce linearmente com o número de navios no cenário.

% A política \textit{greed} também possui custo computacional associado à reordenação dos pontos relevantes. A cada passo, calcula-se a distância entre o VANT e cada ponto candidato, operação com custo $O(n)$, seguida pela ordenação desses pontos por proximidade, que tem custo $O(n \log n)$. No entanto, como essa operação ocorre apenas uma vez por passo de simulação (e não várias vezes como no \textit{Simulated Annealing}), o tempo total de execução é tão próximo de zero que aparenta ser constante, mesmo com o aumento do número de navios.

\subsection{Tempo de Execução}

A Figura~\ref{fig:tempo_execucao} mostra o tempo médio de execução das simulações para cada política, em função do número de navios no cenário. A política \textit{Simulated Annealing} apresenta tempo crescente, enquanto \textit{passiva} e \textit{greed} mantêm tempos quase constantes e significativamente menores.

\begin{figure}[H]
    \centering
    \includegraphics[width=0.7\textwidth]{fig/resultado_tem.png}
    \caption{Tempo médio de execução da simulação por política e quantidade de navios.}
    \label{fig:tempo_execucao}
\end{figure}

O tempo da política \textit{Simulated Annealing} cresce linearmente com o número de navios, pois a cada iteração (controlada por $T_0$, $T_{\text{min}}$, $\beta$ e número de perturbações), calcula-se o custo da rota com complexidade $O(n)$, onde $n$ é o total de pontos a visitar.

A política \textit{greed} também cresce linearmente, pois em cada passo calcula distâncias ($O(n)$) e ordena os candidatos ($O(n \log n)$). No entanto, os tempos absolutos são muito menores (entre 0{,}01 e 0{,}09 segundos), o que faz o gráfico aparentar comportamento constante.

% \subsection{Comparação visual das trajetórias}

% A Figura~\ref{fig:trajetorias_comparacao} apresenta os resultados das simulações com 50 navios para as três políticas de navegação implementadas: (a) \textit{passiva}, (b) \textit{greed} e (c) \textit{Simulated Annealing}. Cada subfigura exibe a rota planejada (waypoints paralelos), o caminho percorrido pelo VANT (trajetória real) e a localização dos navios inspecionados. As imagens destacam as diferenças na forma como a trajetória é ajustada durante o voo, conforme a política adotada.

% \begin{figure}[H]
%     \centering
%     \begin{subfigure}{0.4\textwidth}
%         \centering
%         \includegraphics[width=\linewidth]{fig/passiva.png}
%         \caption{\textit{Passiva}}
%         \label{fig:trajetoria_passiva}
%     \end{subfigure}
%     \hfill
%     \begin{subfigure}{0.4\textwidth}
%         \centering
%         \includegraphics[width=\linewidth]{fig/greed.png}
%         \caption{\textit{Greed}}
%         \label{fig:trajetoria_greed}
%     \end{subfigure}
%     \hfill
%     \begin{subfigure}{0.4\textwidth}
%         \centering
%         \includegraphics[width=\linewidth]{fig/SA.png}
%         \caption{\textit{Simulated Annealing}}
%         \label{fig:trajetoria_sa}
%     \end{subfigure}
%     \caption{Trajetória do VANT para as três políticas de navegação (50 navios).}
%     \label{fig:trajetorias_comparacao}
% \end{figure}

% Na política \textit{passiva}, o VANT segue a rota de referência, inspecionando os navios que encontra ao longo do caminho. Já na política \textit{greed}, o VANT desvia da rota original para inspecionar os navios mais próximos, resultando em uma trajetória mais adaptativa. Por fim, na política \textit{Simulated Annealing}, o VANT também ajusta sua trajetória, mas de forma mais otimizada, buscando minimizar a distância total percorrida. Uma diferença notável é que, na política \textit{greed}, o VANT cruza a própria trajetória algumas vezes, o que não ocorre na política \textit{Simulated Annealing}. Esse comportamento sugere que a política \textit{greed} tende a gerar trajetórias com sobreposição de caminhos, o que pode resultar em maior distância total percorrida. Em contraste, a política \textit{Simulated Annealing} evita esse tipo de cruzamento, produzindo trajetórias mais diretas.

\subsection{Comparação Visual das Trajetórias}

A Figura~\ref{fig:trajetorias_comparacao} mostra os resultados das simulações com 50 navios para as três políticas de navegação. Cada subfigura exibe os waypoints paralelos, a trajetória real do VANT e os navios inspecionados, ilustrando como cada política ajusta o percurso durante o voo.

\begin{figure}[H]
    \centering
    \begin{subfigure}{0.4\textwidth}
        \centering
        \includegraphics[width=\linewidth]{fig/passiva.png}
        \caption{\textit{Passiva}}
        \label{fig:trajetoria_passiva}
    \end{subfigure}
    \hfill
    \begin{subfigure}{0.4\textwidth}
        \centering
        \includegraphics[width=\linewidth]{fig/greed.png}
        \caption{\textit{Greed}}
        \label{fig:trajetoria_greed}
    \end{subfigure}
    \hfill
    \begin{subfigure}{0.4\textwidth}
        \centering
        \includegraphics[width=\linewidth]{fig/SA.png}
        \caption{\textit{Simulated Annealing}}
        \label{fig:trajetoria_sa}
    \end{subfigure}
    \caption{Trajetória do VANT para as três políticas de navegação (50 navios).}
    \label{fig:trajetorias_comparacao}
\end{figure}

Na política \textit{passiva}, o VANT segue rigorosamente a rota de referência. Na \textit{greed}, ele desvia para inspecionar os alvos mais próximos, resultando em uma trajetória mais adaptativa, porém com sobreposição de caminhos. Já na \textit{Simulated Annealing}, a rota também é ajustada, mas de forma mais otimizada, minimizando a distância e evitando cruzamentos. Isso indica maior eficiência da política estocástica em termos de percurso.

% \section{Conclusão}

% Este trabalho apresentou uma abordagem de planejamento dinâmico de rotas para VANTs em missões de patrulha naval, com inserção progressiva de alvos detectados durante o voo. Três políticas de navegação foram implementadas e comparadas: \textit{passiva}, com trajetória fixa; \textit{greed}, que reordena os alvos com base na proximidade; e \textit{Simulated Annealing}, que utiliza uma estratégia estocástica para otimização da sequência de visita. As simulações consideraram diferentes densidades de navios e avaliaram métricas como distância percorrida, cobertura e tempo de execução.

% Os resultados mostraram que a política \textit{greed} obteve, de forma geral, o melhor desempenho, superando o \textit{Simulated Annealing} em cenários com mais de 75 navios. A política baseada em \textit{Simulated Annealing} apresentou vantagens apenas em cenários menos densos, com menor número de alvos. Em cenários com alta densidade de navios, o desempenho das políticas ativas foi impactado pela limitação de autonomia do VANT, que impediu a inspeção completa da área. A comparação evidencia o potencial de estratégias simples, como a política \textit{greed}, em fornecer bons resultados com baixo custo computacional.

% Como trabalhos futuros, propõe-se considerar a dinâmica da aeronave de forma mais realista, incorporando restrições de manobra e aceleração, além de estender o modelo para permitir que os navios estejam em movimento durante a missão. Também seria interessante explorar formas de paralelizar a execução da simulação e da etapa de otimização, permitindo que o replanejamento da rota ocorra em segundo plano sem interromper o avanço do VANT. Além disso, podem ser avaliadas políticas de priorização de alvos, penalidades por cruzamento de trajetória e estratégias de balanceamento entre cobertura e eficiência de percurso. Essas extensões contribuiriam para aproximar ainda mais a simulação de cenários operacionais reais.


\section{Conclusão}

Este trabalho apresentou uma abordagem de planejamento dinâmico de rotas para VANTs em patrulhas navais, com inserção progressiva de alvos detectados durante o voo. Foram comparadas três políticas de navegação: \textit{passiva} (rota fixa), \textit{greed} (priorização por proximidade) e \textit{Simulated Annealing} (otimização estocástica da sequência de visita). As simulações avaliaram diferentes densidades de navios, considerando métricas como distância percorrida, cobertura e tempo de execução.

A política \textit{greed} obteve melhor desempenho geral, especialmente em cenários com mais de 75 navios, combinando bons resultados com baixo custo computacional. Já o \textit{Simulated Annealing} mostrou vantagens em cenários menos densos. Em ambientes mais carregados, a limitação de autonomia do VANT reduziu a eficácia das estratégias ativas.

Como trabalhos futuros, propõe-se: (i) incorporar dinâmica realista do VANT, com restrições de manobra e aceleração; (ii) permitir movimento dos navios; (iii) paralelizar simulação e otimização para replanejamento em tempo real; (iv) investigar funções de custo mais elaboradas, que penalizem cruzamentos de trajetória, priorizem alvos relevantes e equilibrem cobertura com eficiência de percurso. Tais avanços podem tornar a abordagem mais aplicável a cenários operacionais reais.


\bibliographystyle{sbc}
\bibliography{sbc-template}

\end{document}
