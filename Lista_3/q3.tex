\section*{Questão 3: Tempo de mistura}
Considere um processo estocástico que inicia no estado $1$ e a cada instante de tempo incrementa o valor do estado em uma unidade com probabilidade $p$ ou retorna ao estado inicial (estado $1$) com probabilidade $1-p$. No estado $n$ o processo não cresce mais, e se mantém neste estado com probabilidade $p$. Assuma que $n = 10$ e que $p \in \{0.25, 0.5, 0.75\}$.

\begin{enumerate}
    \item Construa a cadeia de Markov deste processo mostrando a matriz de transição de probabilidade em função de $p$.
    \begin{resposta}

        \begin{tikzpicture}[->, >=stealth, node distance=2.2cm,
            every state/.style={circle, draw, minimum size=1.2cm}]
            % States
            \node[state] (1) {1};
            \node[state] (2) [right=of 1] {2};
            \node[state] (3) [right=of 2] {3};
            \node (dots) [right=of 3] {\(\cdots\)};
            \node[state] (10) [right=of dots] {10};

            % Forward transitions
            \draw (1) -- node[above] {\(p\)} (2);
            \draw (2) -- node[above] {\(p\)} (3);
            \draw (3) -- node[above] {\(p\)} (dots);
            \draw (dots) -- node[above] {\(p\)} (10);

            % Return transitions
            \draw[bend left=30] (2) to node[below] {\(1 - p\)} (1);
            \draw[bend left=35] (3) to node[below] {\(1 - p\)} (1);
            \draw[bend left=40] (dots) to node[below] {\(1 - p\)} (1);
            \draw[bend left=45] (10) to node[below] {\(1 - p\)} (1);

            % Self-loops
            \draw (1) edge[loop above] node {\(1 - p\)} (1);
            \draw (10) edge[loop above] node {\(p\)} (10);

        \end{tikzpicture}

        $$
        P_{ij} = \begin{cases}
            1-p & \text{se } j = 1 \\
            p & \text{se } (j = i + 1) \lor (i=j=10)\\
            0 & \text{caso contrário}
        \end{cases}
        $$

    \end{resposta}
    \item Determine numericamente o vão espectral da cadeia de Markov para cada valor de $p$.
    \begin{resposta}

        O vão espectral é dado por:

        $$\delta = 1 - |\lambda_2| $$
        
            onde $\lambda_2$ é o segundo maior autovalor (em módulo) da matriz de transição $P$.

        \newpage
        \begin{itemize}        
            \item \textbf{Para $p = 0{,}25$:}
            \begin{itemize}
                \item Matriz de transição (parcial):
                $$
                \begin{bmatrix}
                0.75 & 0.25 & 0    & \cdots & 0 \\
                0.75 & 0    & 0.25 &        & 0 \\
                \vdots &     &     & \ddots & \vdots \\
                0.75 & 0    & 0    & \cdots & 0.25
                \end{bmatrix}
                $$
                \item Maiores autovalores em módulo:
                $$ |\lambda_1| = 1.0, \quad |\lambda_2| \approx 1.7125 \times 10^{-4} $$
                \item Vão espectral:
                $$ \boxed{\delta = 0.9998287} $$
            \end{itemize}
        
            \item \textbf{Para $p = 0{,}5$:}
            \begin{itemize}
                \item Matriz de transição (parcial):
                $$
                \begin{bmatrix}
                0.5 & 0.5 & 0    & \cdots & 0 \\
                0.5 & 0   & 0.5  &        & 0 \\
                \vdots &     &     & \ddots & \vdots \\
                0.5 & 0   & 0    & \cdots & 0.5
                \end{bmatrix}
                $$
                \item Maiores autovalores em módulo:
                $$ |\lambda_1| = 1.0, \quad |\lambda_2| \approx 3.4251 \times 10^{-4} $$
                \item Vão espectral:
                $$ \boxed{\delta = 0.9996575} $$
            \end{itemize}
        
            \item \textbf{Para $p = 0{,}75$:}
            \begin{itemize}
                \item Matriz de transição (parcial):
                $$
                \begin{bmatrix}
                0.25 & 0.75 & 0    & \cdots & 0 \\
                0.25 & 0    & 0.75 &        & 0 \\
                \vdots &     &     & \ddots & \vdots \\
                0.25 & 0    & 0    & \cdots & 0.75
                \end{bmatrix}
                $$
                \item Maiores autovalores em módulo:
                $$ |\lambda_1| = 1.0, \quad |\lambda_2| \approx 4.3546 \times 10^{-6} $$
                \item Vão espectral:
                $$ \boxed{\delta = 0.9999956} $$
            \end{itemize}
        \end{itemize}
        


    \end{resposta}
    \item Determine numericamente a distribuição estacionária para cada valor de $p$, e indique o estado de menor probabilidade.
    \begin{resposta}
        O estado estacionário $\pi$ de uma cadeia de Markov é um vetor de probabilidade tal que:

        $$ \pi P = \pi $$

        Ou seja, ele é um autovetor à esquerda da matriz de transição $P$ associado ao autovalor $\lambda = 1$. 
        A distribuição estacionária pode ser obtida computando os autovalores e autovetores de $P^\top$, selecionando aquele correspondente a $\lambda = 1$ e normalizando-o para que $\sum_i \pi_i = 1$.

        \begin{itemize}
            \item \textbf{Para $p = 0{,}25$:}
            \begin{itemize}
                \item Distribuição estacionária:
                $$
                \pi = 
                \begin{bmatrix}
                0.75 \\
                0.1875 \\
                0.046875 \\
                0.01171875 \\
                0.00292969 \\
                0.00073242 \\
                0.00018311 \\
                0.00004578 \\
                0.00001144 \\
                0.00000381
                \end{bmatrix}
                $$
            \end{itemize}

            \item \textbf{Para $p = 0{,}5$:}
            \begin{itemize}
                \item Distribuição estacionária:
                $$
                \pi = 
                \begin{bmatrix}
                0.5 \\
                0.25 \\
                0.125 \\
                0.0625 \\
                0.03125 \\
                0.015625 \\
                0.0078125 \\
                0.00390625 \\
                0.00195312 \\
                0.00195312
                \end{bmatrix}
                $$
            \end{itemize}

            \newpage
            \item \textbf{Para $p = 0{,}75$:}
            \begin{itemize}
                \item Distribuição estacionária:
                $$
                \pi = 
                \begin{bmatrix}
                0.25 \\
                0.1875 \\
                0.140625 \\
                0.10546875 \\
                0.07910156 \\
                0.05932617 \\
                0.04449463 \\
                0.03337097 \\
                0.02502823 \\
                0.07508469
                \end{bmatrix}
                $$
            \end{itemize}
        \end{itemize}



    \end{resposta}
    \item Utilizando os dados acima, determine um limitante inferior e superior para o tempo de mistura quando $\epsilon = 10^{-6}$ para cada valor de $p$.
    \begin{resposta}

    \end{resposta}
    \item O que você pode concluir sobre a influência de $p$ no tempo de mistura?
    \begin{resposta}

    \end{resposta}
\end{enumerate}

