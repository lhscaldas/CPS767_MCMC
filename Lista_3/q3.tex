\section*{Questão 3: Tempo de mistura}
Considere um processo estocástico que inicia no estado $1$ e a cada instante de tempo incrementa o valor do estado em uma unidade com probabilidade $p$ ou retorna ao estado inicial (estado $1$) com probabilidade $1-p$. No estado $n$ o processo não cresce mais, e se mantém neste estado com probabilidade $p$. Assuma que $n = 10$ e que $p \in \{0.25, 0.5, 0.75\}$.

\begin{enumerate}
    \item Construa a cadeia de Markov deste processo mostrando a matriz de transição de probabilidade em função de $p$.
    \begin{resposta}

        \begin{tikzpicture}[->, >=stealth, node distance=2.2cm,
            every state/.style={circle, draw, minimum size=1.2cm}]
            % States
            \node[state] (1) {1};
            \node[state] (2) [right=of 1] {2};
            \node[state] (3) [right=of 2] {3};
            \node (dots) [right=of 3] {\(\cdots\)};
            \node[state] (10) [right=of dots] {10};

            % Forward transitions
            \draw (1) -- node[above] {\(p\)} (2);
            \draw (2) -- node[above] {\(p\)} (3);
            \draw (3) -- node[above] {\(p\)} (dots);
            \draw (dots) -- node[above] {\(p\)} (10);

            % Return transitions
            \draw[bend left=30] (2) to node[below] {\(1 - p\)} (1);
            \draw[bend left=35] (3) to node[below] {\(1 - p\)} (1);
            \draw[bend left=40] (dots) to node[below] {\(1 - p\)} (1);
            \draw[bend left=45] (10) to node[below] {\(1 - p\)} (1);

            % Self-loops
            \draw (1) edge[loop above] node {\(1 - p\)} (1);
            \draw (10) edge[loop above] node {\(p\)} (10);

        \end{tikzpicture}

        $$
        P_{ij} = \begin{cases}
            1-p & \text{se } j = 1 \\
            p & \text{se } (j = i + 1) \lor (i=j=10)\\
            0 & \text{caso contrário}
        \end{cases}
        $$

        Matriz de transição (parcial):
                $$
                \begin{bmatrix}
                    (1-p) & p & 0    & \cdots & 0 \\
                    (1-p) & 0    & p &        & 0 \\
                \vdots &     &     & \ddots & \vdots \\
                (1-p) & 0    & 0    & \cdots & p\\
                (1-p) & 0    & 0    & \cdots & p
                \end{bmatrix}
                $$

    \end{resposta}
    \item Determine numericamente o vão espectral da cadeia de Markov para cada valor de $p$.
    \begin{resposta}

        O vão espectral é dado por:
        $$\delta = 1 - |\lambda_2| $$
        
            onde $\lambda_2$ é o segundo maior autovalor (em módulo) da matriz de transição $P$.

        Utilizando um código em Python, com a biblioteca NumPy, podemos calcular os autovalores da matriz de transição $P$ para cada valor de $p$. O código pode ser visto no repositório do Github cujo link encontra-se no fim deste relatório.

        \begin{itemize}        
            \item \textbf{Para $p = 0{,}25$:}
            \begin{itemize}
                \item Matriz de transição (parcial):
                $
                \begin{bmatrix}
                0.75 & 0.25 & 0    & \cdots & 0 \\
                0.75 & 0    & 0.25 &        & 0 \\
                \vdots &     &     & \ddots & \vdots \\
                0.75 & 0    & 0    & \cdots & 0.25 \\
                0.75 & 0    & 0    & \cdots & 0.25
                \end{bmatrix}
                $
                \item Maiores autovalores em módulo:
                $ |\lambda_1| = 1.0, \quad |\lambda_2| \approx 1.7125 \times 10^{-4} $
                \item Vão espectral:
                $ \boxed{\delta = 0.9998287} $
            \end{itemize}
        
            \item \textbf{Para $p = 0{,}5$:}
            \begin{itemize}
                \item Matriz de transição (parcial):
                $
                \begin{bmatrix}
                0.5 & 0.5 & 0    & \cdots & 0 \\
                0.5 & 0   & 0.5  &        & 0 \\
                \vdots &     &     & \ddots & \vdots \\
                0.5 & 0   & 0    & \cdots & 0.5 \\
                0.5 & 0   & 0    & \cdots & 0.5
                \end{bmatrix}
                $
                \item Maiores autovalores em módulo:
                $ |\lambda_1| = 1.0, \quad |\lambda_2| \approx 3.4251 \times 10^{-4} $
                \item Vão espectral:
                $ \boxed{\delta = 0.9996575} $
            \end{itemize}
        
            \item \textbf{Para $p = 0{,}75$:}
            \begin{itemize}
                \item Matriz de transição (parcial):
                $
                \begin{bmatrix}
                0.25 & 0.75 & 0    & \cdots & 0 \\
                0.25 & 0    & 0.75 &        & 0 \\
                \vdots &     &     & \ddots & \vdots \\
                0.25 & 0    & 0    & \cdots & 0.75 \\
                0.25 & 0    & 0    & \cdots & 0.75
                \end{bmatrix}
                $
                \item Maiores autovalores em módulo:
                $ |\lambda_1| = 1.0, \quad |\lambda_2| \approx 4.3546 \times 10^{-6} $
                \item Vão espectral:
                $ \boxed{\delta = 0.9999956} $
            \end{itemize}
        \end{itemize}
        


    \end{resposta}
    \item Determine numericamente a distribuição estacionária para cada valor de $p$, e indique o estado de menor probabilidade.
    \begin{resposta}
        O estado estacionário $\pi$ de uma cadeia de Markov é um vetor de probabilidade tal que:

        $$ \pi P = \pi $$

        Ou seja, ele é um autovetor à esquerda da matriz de transição $P$ associado ao autovalor $\lambda = 1$. 
        A distribuição estacionária pode ser obtida utilizando novamente um código Python com a biblioteca NumPy para computar os autovalores e autovetores de $P^\top$, selecionando aquele correspondente a $\lambda = 1$ e normalizando-o para que $\sum_i \pi_i = 1$.

        \begin{itemize}
            \item \textbf{Para $p = 0{,}25$:}
            \begin{itemize}
                \item Distribuição estacionária:
                $
                \pi = 
                \begin{bmatrix}
                0.75 \\
                0.1875 \\
                0.046875 \\
                0.01171875 \\
                0.00292969 \\
                0.00073242 \\
                0.00018311 \\
                0.00004578 \\
                0.00001144 \\
                \boxed{0.00000381}
                \end{bmatrix}
                $
                \item Estado de menor probabilidade: \textbf{10}, com probabilidade $\boxed{\pi_o = 0.00000381}$.
            \end{itemize}

            \item \textbf{Para $p = 0{,}5$:}
            \begin{itemize}
                \item Distribuição estacionária:
                $
                \pi = 
                \begin{bmatrix}
                0.5 \\
                0.25 \\
                0.125 \\
                0.0625 \\
                0.03125 \\
                0.015625 \\
                0.0078125 \\
                0.00390625 \\
                \boxed{0.00195312} \\
                \boxed{0.00195312}
                \end{bmatrix}
                $
                \item Estado de menor probabilidade: \textbf{9} e \textbf{10}, com probabilidade $\boxed{\pi_o = 0.00195312}$.
            \end{itemize}

            \item \textbf{Para $p = 0{,}75$:}
            \begin{itemize}
                \item Distribuição estacionária:
                $
                \pi = 
                \begin{bmatrix}
                0.25 \\
                0.1875 \\
                0.140625 \\
                0.10546875 \\
                0.07910156 \\
                0.05932617 \\
                0.04449463 \\
                0.03337097 \\
                \boxed{0.02502823} \\
                0.07508469
                \end{bmatrix}
                $
                \item Estado de menor probabilidade: \textbf{9}, com probabilidade $\boxed{\pi_o = 0.02502823}$.
            \end{itemize}
        \end{itemize}

    \end{resposta}

    \newpage
    \item Utilizando os dados acima, determine um limitante inferior e superior para o tempo de mistura quando $\epsilon = 10^{-6}$ para cada valor de $p$.
    \begin{resposta}
        O tempo de mistura $\tau_\epsilon$ representa, de forma aproximada, o número de passos necessários para que a distribuição da cadeia de Markov esteja a uma distância total menor que $\epsilon$ da distribuição estacionária, independentemente do estado inicial.
        
        Ele pode ser estimado por:
        
        $$ \left( \frac{1}{\delta} - 1 \right) \log \frac{1}{2\epsilon} \leq \tau_\epsilon \leq \frac{1}{\delta} \log \frac{1}{\pi_o \epsilon}, $$
        
        onde $\delta$ é o vão espectral da matriz de transição, e $\pi_o = \min_i \pi_i$ é a menor entrada da distribuição estacionária.
        
        \begin{itemize}
            \item \textbf{Para $p = 0{,}25$:}
            \begin{itemize}
                \item $\delta = 0.9998287$ e $\pi_o = 0.00000381$.
                \item Limite inferior estimado:
                $ \boxed{\tau_\epsilon \geq 0.0022476286040300246} $
                \item Limite superior estimado:
                $ \boxed{\tau_\epsilon \leq 26.296663189659515} $
            \end{itemize}
        
            \item \textbf{Para $p = 0{,}5$:}
            \begin{itemize}
                \item $\delta = 0.9996575$ e $\pi_o = 0.00195312$.
                \item Limite inferior estimado:
                $ \boxed{\tau_\epsilon \geq 0.004496027298161922} $
                \item Limite superior estimado:
                $ \boxed{\tau_\epsilon \leq 20.060706093968722} $
            \end{itemize}
        
            \item \textbf{Para $p = 0{,}75$:}
            \begin{itemize}
                \item $\delta = 0.9999956$ e $\pi_o = 0.02502823$.
                \item Limite inferior estimado:
                $ \boxed{\tau_\epsilon \geq 5.7142421447070306 \times 10^{-5}} $
                \item Limite superior estimado:
                $ \boxed{\tau_\epsilon \leq 17.50333771810467} $
            \end{itemize}
        \end{itemize}
    \end{resposta}

    \item O que você pode concluir sobre a influência de $p$ no tempo de mistura?
    \begin{resposta}
        O aumento de $p$ tende a reduzir o tempo de mistura, pois favorece o avanço entre os estados e acelera a aproximação da distribuição estacionária. Já valores menores de $p$ provocam retornos frequentes ao estado inicial, atrasando a convergência.

        Os limites do tempo de mistura refletem esse efeito de formas distintas. O superior diminui com $p$, pois depende inversamente do vão espectral e da menor entrada da distribuição estacionária, ambos crescentes. O inferior, porém, pode crescer inicialmente e depois cair, já que envolve o termo $\left( \frac{1}{\delta} - 1 \right)$, que varia de forma não linear com $p$.
    \end{resposta}
\end{enumerate}

