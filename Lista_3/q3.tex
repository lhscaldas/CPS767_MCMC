\section*{Questão 3: Tempo de mistura}
Considere um processo estocástico que inicia no estado $1$ e a cada instante de tempo incrementa o valor do estado em uma unidade com probabilidade $p$ ou retorna ao estado inicial (estado $1$) com probabilidade $1-p$. No estado $n$ o processo não cresce mais, e se mantém neste estado com probabilidade $p$. Assuma que $n = 10$ e que $p \in \{0.25, 0.5, 0.75\}$.

\begin{enumerate}
    \item Construa a cadeia de Markov deste processo mostrando a matriz de transição de probabilidade em função de $p$.
    \item Determine numericamente o vão espectral da cadeia de Markov para cada valor de $p$.
    \item Determine numericamente a distribuição estacionária para cada valor de $p$, e indique o estado de menor probabilidade.
    \item Utilizando os dados acima, determine um limitante inferior e superior para o tempo de mistura quando $\epsilon = 10^{-6}$ para cada valor de $p$.
    \item O que você pode concluir sobre a influência de $p$ no tempo de mistura?
\end{enumerate}

