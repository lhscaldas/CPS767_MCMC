\section*{Questão 1: Passeios aleatórios enviesados}
Considere um grafo não direcionado $G = (V,E)$ com peso nas arestas, tal que $w_{ij} > 0$ para toda aresta $(i, j) \in E$. Considere um andarilho aleatório que caminha por este grafo em tempo discreto, mas cujos passos são enviesados pelos pesos das arestas. Em particular, a probabilidade do andarilho ir do vértice $i$ para o vértice $j$ é dada por $w_{ij}/W_i$, onde $W_i = \sum_j w_{ij}$ (soma dos pesos das arestas incidentes ao vértice $i \in V$). Temos assim um passeio aleatório enviesado linearmente pelos pesos das arestas.

\begin{enumerate}
    \item Mostre que este passeio aleatório induz uma cadeia de Markov calculando a matriz de transição de probabilidade.
    \begin{resposta}
        
    Conforme o enunciado, temos que a probabilidade de transição do estado $i$ para o estado $j$ é dada por:
    $$ P_{ij} = \frac{w_{ij}}{W_i}, $$
    onde $W_i = \sum_{j} w_{ij}$ é a soma dos pesos das arestas incidentes ao vértice $i$ e $w_{ij}>0$ é o peso da aresta que liga os vértices $i$ e $j$.

    Como $w_{ij} > 0$ para todas as arestas $(i,j) \in E$ e $W_i > 0$, temos que $P_{ij} \geq 0$, sendo estritamente positivo sempre que houver uma aresta entre $i$ e $j$.

    
    Além disso,

    $$ \sum_{j} P_{ij} = \sum_{j} \frac{w_{ij}}{W_i} = \frac{1}{W_i} \sum_{j} w_{ij} = \frac{W_i}{W_i} = 1 $$

    Assim, temos que a soma das probabilidades de transição do estado $i$ para todos os outros estados $j$ é igual a 1, o que caracteriza uma matriz de transição de probabilidade. 
    
    Portanto, o passeio aleatório enviesado \underline{induz uma cadeia de Markov}.
        
    \end{resposta}
    \item Determine a distribuição estacionária desta cadeia de Markov (dica: use o método da inspeção).
    \begin{resposta}
        Uma distribuição estacionária $\pi$ é uma distribuição de probabilidade que satisfaz a seguinte equação:
        $$ \pi P = \pi $$

        Para cada estado $i$ da distribuição estacionária, temos que:
        $$ \pi_i = \sum_{j} \pi_j P_{ji} $$

        Usando o resultado do item anterior, temos que:
        $$ \pi_i = \sum_{j} \pi_j \frac{w_{ji}}{W_j} $$

        Supondo que a distribuição estacionária é proporcional à soma do peso das arestas incidentes ao vértice $j$, ou seja, $\pi_j = Z W_j$, onde $Z>0$ é uma constante, temos que:
        $$ \pi_i = \sum_{j} (Z W_j) \frac{w_{ji}}{W_j} = Z \sum_{j} w_{ji}$$

        Como o grafo é não direcionado, temos que $w_{ij} = w_{ji}$, ou seja:
        $$ \pi_i = Z \sum_{j} w_{ji} = Z \sum_{j} w_{ij} = Z W_i $$

        Porém, sabemos que $\sum_{k} \pi_k = 1$, ou seja:
        $$ \sum_{k} \pi_k = \sum_{k} Z W_k = 1 \quad \Rightarrow \quad Z = \frac{1}{\sum_{k} W_k}$$
    
        Portanto, a distribuição estacionária é dada por:
        $$ \boxed{\pi_i = \frac{W_i}{\sum_{k} W_k}} $$


    \end{resposta}
    \item Determine se esta cadeia de Markov é reversível no tempo.
    \begin{resposta}
        Uma Cadeia de Markov é dita reversível no tempo se, e somente se, satizfaz a equação de fluxo de massa de probabilidade:
        $$ \pi_i P_{ij} = \pi_j P_{ji} $$

        Aplicando a distribuição estacionária que encontramos no item anterior, temos que:
        $$ \pi_i P_{ij} = \frac{W_i}{\sum_{k} W_k} \cdot \frac{w_{ij}}{W_i} = \frac{w_{ij}}{\sum_{k} W_k}  \quad \text{e} \quad \pi_j P_{ji} = \frac{W_j}{\sum_{k} W_k} \cdot \frac{w_{ji}}{W_j} = \frac{w_{ji}}{\sum_{k} W_k}$$
        Como $w_{ij} = w_{ji}$, temos que:
        $$ \frac{w_{ij}}{\sum_{k} W_k} = \frac{w_{ji}}{\sum_{k} W_k} \quad \Rightarrow \quad \boxed{\pi_i P_{ij} = \pi_j P_{ji}} $$
        Portanto, \underline{a cadeia de Markov é reversível} no tempo.
    \end{resposta}
\end{enumerate}
