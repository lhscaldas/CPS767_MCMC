\section*{Questão 1: Sequências binárias restritas}

Considere uma sequência de dígitos binários (0s e 1s) de comprimento $s$. Uma sequência é dita válida se ela não possui 1s adjacentes. Considerando a distribuição uniforme, queremos determinar o valor esperado do número de 1s de uma sequência válida, denotado por $\mu_s$.

\begin{itemize}
  \item Considerando $s = 4$, determine todas as sequências válidas e calcule $\mu_4$.
  \begin{resposta}
    As sequências válidas de comprimento 4 são: 0000, 0001, 0010, 0100, 1000, 0101, 1010. Portanto, temos um total de 7 sequências válidas. O número total de 1s nessas sequências é: $0 + 1 + 1 + 1 + 1 + 2 + 2 = 8$. Assim, $\mu_4 = \frac{8}{7} \approx 1{,}14$.
  \end{resposta}
  \item Construa uma cadeia de Markov sobre o conjunto de sequências válidas, deixando claro como funcionam as transições de estado. Argumente que a cadeia é irredutível.
  \begin{resposta}
  \end{resposta}
  \item Desenhe a cadeia de Markov para o caso de $s = 4$, mostrando todas as transições.
  \begin{resposta}
  \end{resposta}
  \item Mostre como aplicar Metropolis-Hastings para resolver o problema de estimar $\mu_s$. Deixe claro as probabilidades de aceite e o funcionamento do estimador.
  \begin{resposta}
  \end{resposta}
\end{itemize}