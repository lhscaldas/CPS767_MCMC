\section*{Questão 3: Amostrando triângulos}

Considere um grafo conexo qualquer. Desejamos gerar amostras de triângulos deste grafo (cliques de tamanho 3), tal que todo triângulo tenha igual probabilidade de ser amostrado -- ou seja, uma distribuição uniforme sobre o conjunto de triângulos do grafo.

\begin{itemize}
  \item Mostre como gerar amostras de forma direta, utilizando a distribuição uniforme. Dica: pense em amostragem por rejeição. Determine a eficiência desse método.
\begin{resposta}
  O procedimento de amostragem por rejeição consiste em gerar amostras de um espaço amostral maior e, em seguida, rejeitar aquelas que não atendem a um critério específico. Neste caso, o espaço amostral maior é o conjunto de todas as combinações possíveis de 3 vértices do grafo. Para cada combinação, verificamos se os 3 vértices formam um triângulo (ou seja, se estão todos conectados entre si). Se formarem um triângulo, aceitamos a amostra; caso contrário, rejeitamos.

  O procedimento pode ser descrito da seguinte forma:
  \begin{enumerate}
    \item Escolher três vértices distintos $V_1, V_2, V_3 \in V$ uniformemente ao acaso.
    
    \item Verificar se o conjunto $\{V_1, V_2, V_3\}$ forma um triângulo, ou seja, se as três arestas $(V_1, V_2)$, $(V_2, V_3)$ e $(V_1, V_3)$ pertencem ao conjunto de arestas $E$.

    \item Se os vértices formam um triângulo, aceitar a amostra e registrar o triângulo $\{V_1, V_2, V_3\}$.

    \item Caso contrário, rejeitar e voltar ao passo 1.

    \item Repetir o processo até obter o número desejado de amostras.
  \end{enumerate}

  A eficiência do método é dada por:
  $$
  e = \frac{T}{\binom{|V|}{3}}
  $$
  onde $T$ é o número de triângulos no grafo, $|V|$ é a quantidade de vértices e $\binom{|V|}{3}$ representa o número total de trios possíveis de vértices. Assim, a eficiência do método depende da densidade do grafo e do número de triângulos presentes: se o grafo for esparso, a eficiência será baixa; se o grafo for denso, a eficiência será maior.
\end{resposta}

  \item Mostre como gerar amostras utilizando Metropolis-Hastings. Determine os estados da cadeia de Markov, as transições da cadeia base (que deve ser irredutível) e a probabilidade de aceitação na cadeia modificada pelo método Metropolis-Hastings.
  \begin{resposta}
  \end{resposta}
  \item Intuitivamente, discuta quando a abordagem via Metropolis-Hastings é mais eficiente (do ponto de vista computacional) do que a abordagem via amostragem por rejeição.
  \begin{resposta}
  \end{resposta}
\end{itemize}

