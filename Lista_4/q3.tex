\section*{Questão 3: Amostrando triângulos}

Considere um grafo conexo qualquer. Desejamos gerar amostras de triângulos deste grafo (cliques de tamanho 3), tal que todo triângulo tenha igual probabilidade de ser amostrado — ou seja, uma distribuição uniforme sobre o conjunto de triângulos do grafo.

\begin{itemize}
  \item Mostre como gerar amostras de forma direta, utilizando a distribuição uniforme. Dica: pense em amostragem por rejeição. Determine a eficiência desse método.
  \begin{resposta}
  \end{resposta}
  \item Mostre como gerar amostras utilizando Metropolis-Hastings. Determine os estados da cadeia de Markov, as transições da cadeia base (que deve ser irredutível) e a probabilidade de aceitação na cadeia modificada pelo método Metropolis-Hastings.
  \begin{resposta}
  \end{resposta}
  \item Intuitivamente, discuta quando a abordagem via Metropolis-Hastings é mais eficiente (do ponto de vista computacional) do que a abordagem via amostragem por rejeição.
  \begin{resposta}
  \end{resposta}
\end{itemize}

