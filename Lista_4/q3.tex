\section*{Questão 3: Amostrando triângulos}

Considere um grafo conexo qualquer. Desejamos gerar amostras de triângulos deste grafo (cliques de tamanho 3), tal que todo triângulo tenha igual probabilidade de ser amostrado -- ou seja, uma distribuição uniforme sobre o conjunto de triângulos do grafo.

\begin{itemize}
  \item Mostre como gerar amostras de forma direta, utilizando a distribuição uniforme. Dica: pense em amostragem por rejeição. Determine a eficiência desse método.
\begin{resposta}
  O procedimento de amostragem por rejeição consiste em gerar amostras de um espaço amostral maior e, em seguida, rejeitar aquelas que não atendem a um critério específico. Neste caso, o espaço amostral maior é o conjunto de todas as combinações possíveis de 3 vértices do grafo. Para cada combinação, verificamos se os 3 vértices formam um triângulo (ou seja, se estão todos conectados entre si). Se formarem um triângulo, aceitamos a amostra; caso contrário, rejeitamos.

  O procedimento pode ser descrito da seguinte forma:
  \begin{enumerate}
    \item Escolher três vértices distintos $u, v, w \in V$ uniformemente ao acaso.
    
    \item Verificar se o conjunto $\{u, v, w\}$ forma um triângulo, ou seja, se as três arestas $(u, v)$, $(v, w)$ e $(u, w)$ pertencem ao conjunto de arestas $E$.

    \item Se os vértices formam um triângulo, aceitar a amostra e registrar o triângulo $\{u, v, w\}$.

    \item Caso contrário, rejeitar e voltar ao passo 1.

    \item Repetir o processo até obter o número desejado de amostras.
  \end{enumerate}

  A eficiência do método é dada por:
  $$
  e = \frac{T}{\binom{|V|}{3}}
  $$
  onde $T$ é o número de triângulos no grafo, $|V|$ é a quantidade de vértices e $\binom{|V|}{3}$ representa o número total de trios possíveis de vértices. Assim, a eficiência do método depende da densidade do grafo e do número de triângulos presentes: se o grafo for esparso, a eficiência será baixa; se o grafo for denso, a eficiência será maior.
\end{resposta}

  \item Mostre como gerar amostras utilizando Metropolis-Hastings. Determine os estados da cadeia de Markov, as transições da cadeia base (que deve ser irredutível) e a probabilidade de aceitação na cadeia modificada pelo método Metropolis-Hastings.
  \begin{resposta}
    Podemos criar uma Cadeia de Markov na qual os estados são caminhos de tamanho 3, ou seja, conjuntos de 3 vértices $\{u, v, w\}$, onde, pelo menos $uv$ e $vw$ são arestas do grafo.

    A regra de transição da cadeia base pode ser definida como a remoção de um dos vértices da ponta ($u$ ou $w$) e a adição de um novo vértice que esteja conectado ao vértice da outra ponta. Assim, um estado $\{u, v, w\}$ pode transicionar para $\{v, w, x\}$ se escolhermos remover $u$ e escolher $x$ tal que é vizinho de $w$, ou pode transicionar para $\{x, u, v\}$, se escolhermos remover $w$ e escolher $x$ tal que é vizinho de $u$.
    A cadeia é irredutível porque, a partir de qualquer caminho de comprimento 3, podemos alcançar qualquer outro caminho de comprimento 3 removendo um vértice de uma ponta e adicionando outro conectado na outra ponta.
    
    A probabilidade de transição entre dois estados é dada por:
    $$ P_{s \rightarrow s'} = \frac{1}{d-1} $$
    onde $d$ é o grau do vértice da ponta contrária a que foi removida ($u$ ou $w$) do estado atual menos 1 (pois não podemos escolher o vértice $v$ que já está no meio do caminho de comprimento 2 e irá para uma das pontas no próximo estado).

    A probabilidade de transição na cadeia modificada pelo método Metropolis-Hastings é dada por:
    $$ P'_{s \rightarrow s'} = \frac{1}{d-1} \min\left(1, \frac{d-1}{d'} \right) $$
    onde $d'$ é o grau do novo vértice que será adicionado ao caminho no estado $s'$.

    É fácil ver que nem todos os estados da Cadeia de Markov são necessariamente triângulos, mas todos os triângulos são alcançados a partir de algum estado da cadeia. Então podemos condicionar o processo de amostragem para que apenas triângulos sejam aceitos. Como a Cadeia de Markov modificada possui distribuição estacionária uniforme, a distribuição dessa amostragem condicional também será uniforme.

    O procedimento de amostragem pode ser descrito da seguinte forma:
    \begin{enumerate}
      \item Escolher um caminho de comprimento 2 inicial $s=\{u, v, w\}$ uniformemente ao acaso.
      \item Executar o procedimento abaixo:
      \begin{enumerate}
        \item Escolher com probabilidade $\frac{1}{2}$ remover o primero vértice ($u$) ou o terceiro vértice ($w$).
        \item Escolher com probabilidade uniforme $\frac{1}{d-1}$ um novo vértice $x$ que seja vizinho do vértice da outra ponta (ou seja, $u$ se $w$ foi removido ou $w$ se $u$ foi removido).
        \item Gerar um número aleatório $r$ uniformemente no intervalo $[0, 1]$.
        \item Se $r <  \min\left(1, \frac{d-1}{d'} \right)$, aceitar a amostra e registrar $s'$, senão, repetir o estado anterior fazendo $s'=s$.
      \end{enumerate}
      \item Repetir os passos $a - d$ por $\tau_\epsilon$ iterações, onde $\tau_\epsilon$ é o tempo de mistura da Cadeia de Markov.
      \item Verificar se o último estado $s'$ é um triângulo, ou seja, se as arestas $(u', v')$, $(v', w')$ e $(u', w')$ pertencem ao conjunto de arestas $E$. Caso sim, aceitar a amostra e registrar o triângulo $s'$, caso o contrário, continuar repetindo os passos $a - d$ até obter um triângulo.
      \item Repetir os passos $2 - 4$ até obter o número desejado de amostras de triângulos.
    \end{enumerate}

  \end{resposta}
  \item Intuitivamente, discuta quando a abordagem via Metropolis-Hastings é mais eficiente (do ponto de vista computacional) do que a abordagem via amostragem por rejeição.
  \begin{resposta}
    A abordagem via Metropolis-Hastings tende a ser mais eficiente que a amostragem por rejeição quando o grafo é esparso, ou seja, quando o número de triângulos \(T\) é pequeno em relação ao número total de combinações de 3 vértices \(\binom{|V|}{3}\). Nesse cenário, a amostragem por rejeição desperdiça muitas tentativas, pois a chance de um trio aleatório de vértices formar um triângulo é muito baixa. Como resultado, o número de rejeições cresce rapidamente, tornando o método ineficiente em tempo de execução.

    Por outro lado, a abordagem via Metropolis-Hastings opera sobre caminhos de comprimento 2, ou seja, trios de vértices onde duas das três conexões já existem por construção. Isso significa que, em cada iteração, a cadeia considera apenas trios parcialmente conectados, nos quais falta apenas uma aresta para formar um triângulo. Assim, a chance de o trio atual ser um triângulo é muito maior do que na amostragem por rejeição, que escolhe vértices completamente ao acaso. Dessa forma, mesmo rejeitando trios que não formam triângulos, o processo explora mais eficientemente o espaço de triângulos do grafo.
\end{resposta}

\end{itemize}

