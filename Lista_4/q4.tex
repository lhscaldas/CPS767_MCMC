\section*{Questão 4: Quebrando o código}

Você encontrou uma mensagem cifrada com o código de substituição (neste código, cada letra é mapeada em outra letra de forma bijetiva). Você deseja encontrar a chave do código para ler a mensagem. Repare que a chave é um mapeamento $\sigma$ entre as letras, por exemplo $\sigma(a) = x$, $\sigma(b) = h$, $\sigma(c) = e$, etc. Considere uma função $f : \Omega \rightarrow [0,1]$ que avalia a capacidade de uma pessoa entender a mensagem cifrada dado um mapeamento $\sigma \in \Omega$. Repare que $f(\sigma) = 1$ significa que é possível entender por completo a mensagem decifrada com o mapeamento $\sigma$, e $f(\sigma) = 0$ se o mapeamento não revela nenhuma informação sobre a mensagem. Mostre como a técnica de Simulated Annealing pode ser utilizada para ler a mensagem cifrada. Mostre todos os passos necessários para aplicar a técnica neste problema (não é necessário implementar).
\begin{resposta}
\end{resposta}
